%%
%% This is file `template-6s.tex',
%% generated with the docstrip utility.
%%
%% The original source files were:
%%
%% template.raw  (with options: `6s')
%% 
%% Template for the LaTeX class aipproc.
%% 
%% (C) 1998,2000,2001 American Institute of Physics and Frank Mittelbach
%% All rights reserved
%% 
%%
%% $Id: template.raw,v 1.12 2005/07/06 19:22:14 frank Exp $
%%

%%%%%%%%%%%%%%%%%%%%%%%%%%%%%%%%%%%%%%%%%%%%
%% Please remove the next line of code if you
%% are satisfied that your installation is
%% complete and working.
%%
%% It is only there to help you in detecting
%% potential problems.
%%%%%%%%%%%%%%%%%%%%%%%%%%%%%%%%%%%%%%%%%%%%

\input{aipcheck}

%%%%%%%%%%%%%%%%%%%%%%%%%%%%%%%%%%%%%%%%%%%%
%% SELECT THE LAYOUT
%%
%% The class supports further options.
%% See aipguide.pdf for details.
%%
%%%%%%%%%%%%%%%%%%%%%%%%%%%%%%%%%%%%%%%%%%%%

\documentclass[
    ,final            % use final for the camera ready runs
%%  ,draft            % use draft while you are working on the paper
%%  ,numberedheadings % uncomment this option for numbered sections
%%  ,                 % add further options here if necessary
  ]
  {aipproc}

\layoutstyle{6x9}

%%%%%%%%%%%%%%%%%%%%%%%%%%%%%%%%%%%%%%%%%%%%
%% FRONTMATTER
%%%%%%%%%%%%%%%%%%%%%%%%%%%%%%%%%%%%%%%%%%%%

\newcommand{\gee}{$\Gamma_{ee}$}
\newcommand{\ups}{$\Upsilon$}
\newcommand{\us}{$\Upsilon(1S)$}
\newcommand{\uss}{$\Upsilon(2S)$}
\newcommand{\usss}{$\Upsilon(3S)$}
\newcommand{\ee}{$e^+e^-$}
\newcommand{\mm}{$\mu^+\mu^-$}
\newcommand{\tautau}{$\tau^+\tau^-$}
\newcommand{\ellell}{$\ell^+\ell^-$}
\newcommand{\pipi}{$\pi^+\pi^-$}
\newcommand{\PM}{$\pm$}
\newcommand{\inv}{$^{-1}$}
\newcommand{\bmm}{${\mathcal B}_{\mu\mu}$}
\newcommand{\btt}{${\mathcal B}_{\tau\tau}$}
\newcommand{\geehadtot}{\Gamma_{ee}\Gamma_{\mbox{had}}/\Gamma_{\mbox{tot}}}
\newcommand{\pvis}{P_{\mbox{vis}}}
\newcommand{\ppass}{P_{\mbox{pass given vis}}}
\newcommand{\ehtrig}{\epsilon_{\mbox{htrig}}}
\newcommand{\ecuts}{\epsilon_{\mbox{cuts}}}
\newcommand{\chired}{\chi^2_{\mbox{red}}}

\begin{document}

\title{Di-electron Widths \\ of the \boldmath $\Upsilon(1S,\,2S,\,3S)$ Resonances}

\classification{14.40.Nd, 12.20.Fv, 13.25.Gv}
\keywords      {Upsilon bottomonium di-electron di-lepton partial width resonance scan}

\author{J.~Pivarski}{
  address={Cornell University, Ithaca, New York 14853}
}


\begin{abstract}
We determine the di-electron widths of the \us, \uss, and \usss\
resonances with better than 2\% precision by integrating the
cross-section of $e^+e^- \to \Upsilon$ over the \ee\ center-of-mass
energy.  Using \ee\ energy scans of the \ups\ resonances at the
Cornell Electron Storage Ring and measuring \ups\ production with the
CLEO detector, we find di-electron widths of
%
1.252 \PM\ 0.004 ($\sigma_{\mbox{stat}}$) \PM\ 0.019 ($\sigma_{\mbox{syst}}$) keV,
0.581 \PM\ 0.004 \PM\ 0.009 keV, and
0.413 \PM\ 0.004 \PM\ 0.006 keV for the \us, \uss, and \usss,
respectively.
\end{abstract}

\maketitle

%%%%%%%%%%%%%%%%%%%%%%%%%%%%%%%%%%%%%%%%%%%%
%% MAINMATTER
%%%%%%%%%%%%%%%%%%%%%%%%%%%%%%%%%%%%%%%%%%%%

The widths of the \ups\ mesons, $b\bar{b}$ bound states discovered in
1977 \cite{discovery}, are related to the quark-antiquark spatial wave
function at the origin \cite{wavefunction}.  Recently, these widths
have been recognized as a testing ground for QCD lattice gauge theory
calculations \cite{lattice}.  Improvements in the lattice
calculations, such as the avoidance of the quenched approximation
\cite{unquenched}, provide an incentive for more accurate experimental
tests.  The di-electron widths (\gee) of the \us, \uss, and \usss\
have previously been measured with precisions of 2.2\%, 4.2\%, and
9.4\%, respectively \cite{pdg}.  Validation of the lattice
calculations at an accuracy of a few percent will increase confidence
in similar calculations used to extract important weak-interaction
parameters from data.  In particular, \gee\ and $f_D$ \cite{fd}
provide complementary tests of the calculation of $f_B$, which is used
to determine the CKM parameter $V_{td}$.

At PANIC05, we presented preliminary measurements of \gee\ which have
since been supplanted by public final results, available in
\cite{prl}.  This reference covers the same experimental issues as our
presentation.  More detail will soon be available in
\cite{thesis}.  The theoretical lattice prediction we referred to in
our presentation has since been published in
\cite{lattice} (near the end).  To avoid repetition, we refer the reader to these documents.

%%%%%%%%%%%%%%%%%%%%%%%%%%%%%%%%%%%%%%%%%%%%%%%%
%% BACKMATTER
%%%%%%%%%%%%%%%%%%%%%%%%%%%%%%%%%%%%%%%%%%%%%%%%

%% \begin{theacknowledgments}
%% We gratefully acknowledge the effort of the CESR staff 
%% in providing us with excellent luminosity and running conditions.
%%  This work was supported by 
%% the A.P.~Sloan Foundation,
%% the National Science Foundation,
%% and the U.S. Department of Energy.
%% \end{theacknowledgments}

%%%%%%%%%%%%%%%%%%%%%%%%%%%%%%%%%%%%%%%%%%%
%% The following lines show an example how to produce a bibliography
%% without the help of the BibTeX program. This could be used instead
%% of the above.
%%%%%%%%%%%%%%%%%%%%%%%%%%%%%%%%%%%%%%%%%%%

\begin{thebibliography}{9}
\bibitem{discovery}
S.W.~Herb {\it et al.},
%``Observation Of A Dimuon Resonance At 9.5-Gev In 400-Gev Proton - Nucleus
%Collisions,''
Phys.\ Rev.\ Lett.\  {\bf 39}, 252 (1977).
%%CITATION = PRLTA,39,252;%%

\bibitem{wavefunction}
R.~Van Royen and V.F.~Weisskopf,
%``Hadron Decay Processes And The Quark Model,''
Nuovo Cim.\ A {\bf 50}, 617 (1967)
[Erratum ibid.\ {\bf 51}, 583 (1967)].
%%CITATION = NUCIA,A50,617;%%

\bibitem{lattice}
%% A.~Gray {\it et al.} %, I.~Allison, C.T.H.~Davies, E.~Gulez, G.P.~Lepage, J.~Shigemitsu and M.~Wingate,
%% %``The Upsilon spectrum and m(b) from full lattice QCD,''
%% arXiv:hep-lat/0507013.
%% %%CITATION = HEP-LAT 0507013;%%
A.~Gray, I.~Allison, C.~T.~H.~Davies, E.~Gulez, G.~P.~Lepage, J.~Shigemitsu and M.~Wingate,
%``The Upsilon spectrum and m(b) from full lattice QCD,''
Phys.\ Rev.\ D {\bf 72}, 094507 (2005).
% [arXiv:hep-lat/0507013].
%%CITATION = HEP-LAT 0507013;%%

\bibitem{unquenched}
C.T.H.~Davies {\it et al.}  (HPQCD Collaboration),
%``High-precision lattice QCD confronts experiment,''
Phys.\ Rev.\ Lett.\  {\bf 92}, 022001 (2004).
%[arXiv:hep-lat/0304004].
%%CITATION = HEP-LAT 0304004;%%

\bibitem{pdg}
S.~Eidelman {\it et al.}  (Particle Data Group),
%``Review of particle physics,''
Phys.\ Lett.\ B {\bf 592}, 1 (2004).
%%CITATION = PHLTA,B592,1;%%

\bibitem{fd}
%% M.~Artuso {\it et al.}  (CLEO Collaboration),
%% %``Improved measurement of B(D+ $\to$ mu+ nu) and the pseudoscalar decay
%% %constant f(D+),''
%% arXiv:hep-ex/0508057, to be published in Phys.\ Rev.\ Lett.
%% %%CITATION = HEP-EX 0508057;%%
M.~Artuso {\it et al.}  (CLEO Collaboration),
Phys.\ Rev.\ Lett.\ {\bf 95}, 251801 (2005).

\bibitem{prl}
J.~L.~Rosner  [CLEO Collaboration],
%``Di-electron widths of the Upsilon(1S,2S,3S) resonances,''
arXiv:hep-ex/0512056.
%%CITATION = HEP-EX 0512056;%%

\bibitem{thesis}
J.~Pivarski, Cornell University, Ph.D.~thesis (unpublished).

\end{thebibliography}

\end{document}

\endinput
%%
%% End of file `template-6s.tex'.
