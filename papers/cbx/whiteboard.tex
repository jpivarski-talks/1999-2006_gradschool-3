\documentclass[12pt]{article}

\oddsidemargin  -0.5 cm
\evensidemargin 0.0 cm
\textwidth      6.5in
\headheight     0.0in
\topmargin      -1 cm
\textheight=9.0in

\begin{document}

\section*{The question on my whiteboard}

For what $\sigma_E$ is
\begin{equation}
  S(\Delta_E) = \sum_{i=1}^{30} \frac{(\sigma_{i,1} - \sigma_{i,2})^2}{\Delta_{i,1}^2 + \Delta_{i,2}^2 + (\Delta_E \times f')^2} \cdot \frac{1}{N-1} = 1
\end{equation}
(where $\sigma_{i,1} \pm \Delta_{i,1}$, $\sigma_{i,2} \pm
\Delta_{i,2}$ are cross-sections and $f'$ is the derivative of the
lineshape)?

The answer to this question is $S(0) = 1.66$, $S(0.15\mbox{ MeV}) = 0.99$.

I thought this might be a little unfair, since this formula assumes
that the beam energy program reported the same beam energy for the
pair of measurements: it doesn't subtract $\Delta_{E\mbox{,
expected}}$.  I applied this formula to the 15 pairs that {\it did}
report the same beam energy, and recieved the same result: $S'(0) =
1.22$, $S'(0.15\mbox{ MeV}) = 0.97$.

It also differs in that I first converted all pairs to energy shifts
and then tried to extract a limit, rather than 







\end{document}
