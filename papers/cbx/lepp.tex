% Prepared by Karl Ecklund September 2002
% Questions, Improvements, Comments to kme and CLEOAC
%
% LATEX 2e Template for CLEO Papers
% YOU MUST USE REVTEX4 and latex 2e
%
% Checklist for Paper Drafts:
% ---------------------------
% 0) Use appropriate \documetclass line as indicated below
% 1) Draft Number: Use latest CBX number, append A for the first vote
%                                (B,C,... for subsequent if any votes)
% 2) Don't use CLNS or CLEO numbers - this happens after your vote.
% 3) Title; use \\ to break title over several lines.
% 4) Abstract
% 5) For the Author list use CLEO Collaboration only
% 6) Body
% 
% Checklist for Journal Submissions:
% ----------------------------------
% 0) Use appropriate \documetclass line as indicated below
%    - For CLNS, hep-ex, preprints, conf. paper use CLNS version
%    - For PRL submission use PRL version
%    - For PRD submission use PRD version 
%      AND use \author{(CLEO Collaboration)} instead of \collaboration{CLEO}
%      SEE PRD_SPECIAL_CHANGEME in author list during step 5 below
% 1) CLEO paper number (from CLEOAC)
% 2) CLNS preprint number (from CLEOAC)
% 3) Title; use \\ to break title over several lines.
% 4) Abstract
% 5) Author list (from CLEOAC)
% 6) PACS codes
% 7) Body
% 8) Add acknowledgments
% 9) Hardcode the \date when ready to submit to journal and hep-ex.
%
% Checklist for Conference Papers:
% --------------------------------
% 0) Use appropriate \documetclass line as indicated below
%    - For conf. paper use CLNS version
% 1) CLEO conference paper number (from CLEOAC) (don't use CLNS)
% 2) Title; use \\ to break title over several lines.
% 3) To \thanks after title, add appropriate conference info.
% 4) Abstract
% 5) Author list (from CLEOAC)
% 6) Body
% 7) Add acknowledgments
% 8) Hardcode the \date when ready to submit to conference and hep-ex.
% 
% KNOWN PROBLEMS with template and REVTEX4
% - can't have \collaboration in PRD style grouped author list
%   Using \author{(CLEO Collaboration)} instead
% - can't make abstract appear before full author list ala CLNS notes
%   Abandoning this as the format for CLNS.

%%%%%%%%%%%%%%%%%%%%%%%%%%%%%%%%%%%%%%%%%%%%%%%%%%%%%%%%%%%%%%%%%%%%%%
% Select one of the \documentclass lines below for your paper
%%%%%%%%%%%%%%%%%%%%%%%%%%%%%%%%%%%%%%%%%%%%%%%%%%%%%%%%%%%%%%%%%%%%%%
%%%%%%%%%%%%%% Use for CLNS preprint (hep-ex) and Paper Drafts
\documentclass[aps,prd,preprint,superscriptaddress,tightenlines,nofootinbib,floatfix]{revtex4}

%%%%%%%%%%%%%% Use for PRL
%\documentclass[aps,prl,twocolumn,superscriptaddress,showpacs]{revtex4}

%%%%%%%%%%%%%% Use for PRD submission
%\documentclass[aps,prd,preprint,nopreprintnumbers,nofootinbib,showpacs]{revtex4}
%%%%%%%%%%%%%% Use for PRD formatting tables and figures in 2 column
%\documentclass[aps,prd,twocolumn,nofootinbib,showpacs]{revtex4}

\usepackage{graphicx}% Include figure files
\usepackage{dcolumn}% Align table columns on decimal point
\usepackage{bm}% bold math
\usepackage{multirow}% multirow table entries
\usepackage{amsmath}

\begin{document}

%-Definitions-----------------------------------------------------------
\newcommand{\barb}{\bar{B}}
%\newcommand{\bbbar}{B\bar{B}}
%\newcommand{\bm}{B^-}
\newcommand{\bob}{\bar{B}^0}
\newcommand{\bo}{B^0}
\newcommand{\bp}{B^+}
\newcommand{\dkpipiz}{D^0 \to K^-\pi^+\pi^0}
\newcommand{\dkpi}{D^0 \to K^-\pi^+}
\newcommand{\dkrho}{D^0 \to K^- \rho^+}
\newcommand{\dobar}{\overline{D}^0}
\newcommand{\etal}{{\it et al.}}
\newcommand{\F}{$\cal F$}
%\newcommand{\gev}{\ \rm GeV}
\newcommand{\kpipiz}{K^-\pi^+\pi^0}
%\newcommand{\kpi}{K^-\pi^+}
\newcommand{\krho}{K^- \rho^+}
%\newcommand{\mev}{\ \rm MeV}
\newcommand{\piz}{\pi^0}
\newcommand{\ra}{{\rightarrow}}
\newcommand{\Ufs}{\Upsilon(4S)}

\newcommand{\about}		{\mbox{$\sim$}}
\newcommand{\aerr}[3]   {\mbox{${{#1}^{+ #2}_{- #3}}$}}
\newcommand{\amp}    	{\mbox{${\cal A}$}}
\newcommand{\avcb}		{\mbox{$|V_{cb}|$}}
\newcommand{\avtb}		{\mbox{$|V_{tb}|$}}
\newcommand{\avtd}		{\mbox{$|V_{td}|$}}
\newcommand{\avts}		{\mbox{$|V_{ts}|$}}
\newcommand{\avub}		{\mbox{$|V_{ub}|$}}
\newcommand{\avud}		{\mbox{$|V_{ud}|$}}
\newcommand{\avus}		{\mbox{$|V_{us}|$}}
\newcommand{\BBbar}     {\mbox{$B\bar B$}}
\newcommand{\bbbar}     {\mbox{$B\bar B$}}
\newcommand{\bc}		{\mbox{$b\to c$}}
\newcommand{\bdb}		{\mbox{$\bar B^0_d $}}
\newcommand{\bd}		{\mbox{$B^0_d $}}
\newcommand{\berr}[2]   {\mbox{${{}^{+ #1}_{- #2}}$}}
\newcommand{\bmeson}	{\mbox{$B$}}
\newcommand{\bqb}		{\mbox{$\bar B^0_q $}}
\newcommand{\bq}		{\mbox{$B^0_q $}}
\newcommand{\branch}    {\mbox{${\cal B}$}}
\newcommand{\bsb}		{\mbox{$\bar B^0_s$}}
\newcommand{\bs}		{\mbox{$B^0_s$}}
\newcommand{\bu}		{\mbox{$b\to u$}}
\newcommand{\cerr}[4]   {\mbox{${{}^{+ #1}_{- #2}{}^{+ #3}_{- #4}}$}}
\newcommand{\chisq}		{\mbox{$\chi^2$}}
\newcommand{\cosB}		{\mbox{${\cos\theta_{\rm B}}$}}
\newcommand{\cossph}	{\mbox{$\cos \theta_{\rm sph}$}}
\newcommand{\costhr}	{\mbox{$\cos \theta_{\rm thr}$}}
\newcommand{\dedx}		{\mbox{$dE/dx$}}
\newcommand{\derr}[5]   {\mbox{${{#1}^{+ #2}_{- #3}{}^{+ #4}_{- #5}}$}}
\newcommand{\de}		{\mbox{$\Delta E$}}
\newcommand{\dzerobar}	{\mbox{$\overline {D^0}$}}
\newcommand{\dzero}		{\mbox{${D^0}$}}
\newcommand{\ebeam}		{\mbox{$E_{\rm beam}$}}
\newcommand{\eb}		{\mbox{$E_b$}}
\newcommand{\eeqq}		{\mbox{$e^+e^-\to\qqb$}}
\newcommand{\ee}		{\mbox{$e^+e^-$}}
%\newcommand{\etal}		{\mbox{${\it et al}$}}
\newcommand{\expt}		{\mbox{$_{\rm expt}$}}
\newcommand{\fbinv}		{\mbox{${\rm fb}^{-1}$}}
%\newcommand{\fisher}    {\mbox{${\cal F}$}}
\newcommand{\fisher}    {\mbox{$x_{\cal F}$}}
\newcommand{\gev}		{\mbox{${\rm ~GeV}$}}
\newcommand{\hh}		{\mbox{$h^+h^-$}}
\newcommand{\hpm}		{\mbox{$h^\pm$}}
%\newcommand{\implies}	{\mbox{${\Longrightarrow}$}}
\newcommand{\jimexp}[1]	{\mbox{${\rm e}^{{#1}}$}}
\newcommand{\kk}		{\mbox{KK}}
\newcommand{\Kpi}		{\mbox{$K\pi$}}
\newcommand{\kpi}		{\mbox{$\Kpi$}}
\newcommand{\kpz}		{\mbox{$K^+\pi^0$}}
\newcommand{\KP}		{\mbox{$K\pi$}}
\newcommand{\ksp}		{\mbox{$K^0_S\pi^+$}}
\newcommand{\ks}		{\mbox{$K^0_S$}}
\newcommand{\kz}		{\mbox{$K^0$}}
\newcommand{\kzb}		{\mbox{$\overline{K^0}$}}
\newcommand{\like}    	{\mbox{${\cal L}$}}
\newcommand{\Lp}		{\mbox{$\Lambda \bar p$}}
\newcommand{\lum}    	{\mbox{${\cal L}$}}
\newcommand{\mb}		{\mbox{$M_B$}}
\newcommand{\mev}		{\mbox{${\rm MeV}$}}
\newcommand{\micron}	{\mbox{$~\mu{\rm m}$}}
\newcommand{\mm}		{\mbox{$\mu^+\mu^-$}}
\newcommand{\model}		{\mbox{$_{\rm mod}$}}
\newcommand{\nbb}		{\mbox{$N_{B\bar B}$}}
\newcommand{\nbinv}		{\mbox{${\rm nb}^{-1}$}}
\newcommand{\pb}        {\mbox{$p_B$}}
\newcommand{\pbinv}		{\mbox{${\rm pb}^{-1}$}}
\newcommand{\pdf}		{\mbox{${PDF}$}}
\newcommand{\pdfs}		{\mbox{${PDF}{\rm s}$}}
\newcommand{\pidkpi}	{\mbox{$\Delta_{K\pi}$}}
\newcommand{\pidkp}		{\mbox{$\Delta_{Kp}$}}
\newcommand{\pid}		{\mbox{$\Delta_{PID}$}}
\newcommand{\pipi}		{\mbox{$\Pipi$}}
\newcommand{\Pipi}		{\mbox{$\pi\pi$}}
\newcommand{\power}[1]  {\mbox{${\times 10^{#1}}$}}
\newcommand{\ppz}		{\mbox{$\pi^+\pi^0$}}
\newcommand{\pvec}		{\mbox{$\vec{p}$}}
\newcommand{\pz}		{\mbox{$\pi^0$}}
\newcommand{\qqb}		{\mbox{$q\bar q$}}
\newcommand{\qq}		{\mbox{${q\bar q}$}}
\newcommand{\stat}		{\mbox{$_{\rm stat}$}}
\newcommand{\syst}		{\mbox{$_{\rm syst}$}}
\newcommand{\theo}		{\mbox{$_{\rm theo}$}}
\newcommand{\upsi}		{\mbox{$\Upsilon$({\rm 4S})}}
\newcommand{\vcb}		{\mbox{$V_{cb}$}}
\newcommand{\vcd}		{\mbox{$V_{cd}$}}
\newcommand{\vcs}		{\mbox{$V_{cs}$}}
\newcommand{\vtb}		{\mbox{$V_{tb}$}}
\newcommand{\vtd}		{\mbox{$V_{td}$}}
\newcommand{\vts}		{\mbox{$V_{ts}$}}
\newcommand{\vub}		{\mbox{$V_{ub}$}}
\newcommand{\vud}		{\mbox{$V_{ud}$}}
\newcommand{\vus}		{\mbox{$V_{us}$}}
\newcommand{\zhat}		{\mbox{$\hat{\bf z}$}}

\newcommand{\ppbar}		{\mbox{${p\bar p}$}}
\newcommand{\pL}		{\mbox{${p\bar\Lambda}$}}
\newcommand{\LL}		{\mbox{${\Lambda\bar\Lambda}$}}

\newcommand{\mpdf}    	{\mbox{${\cal M}$}}
\newcommand{\epdf}    	{\mbox{${\cal E}$}}
\newcommand{\fpdf}    	{\mbox{${\cal F}$}}
\newcommand{\cpdf}    	{\mbox{${\cal C}$}}
\newcommand{\dkmode}	{\mbox{${\mu}$}}
\newcommand{\contrib}	{\mbox{${\kappa}$}}
\newcommand{\mc}    	{\mbox{${}_{\dkmode\contrib}$}}
\newcommand{\nk}    	{\mbox{${n}_{\contrib}$}}

\newcommand{\G}		{\mbox{${G}$}}
\newcommand{\GG}	{\mbox{${\cal G}$}}
\newcommand{\ARG}	{\mbox{${A}$}}
\newcommand{\LIN}	{\mbox{${L}$}}
\newcommand{\BW}	{\mbox{${\cal R}$}}
\newcommand{\FI}	{\mbox{${F_0}$}}
\newcommand{\DG}	{\mbox{${a\G_1+b\G_2}$}}
\newcommand{\GGG}	{\mbox{${a\G+b\GG}$}}
\newcommand{\DGG}	{\mbox{${a\GG_1+b\GG_2}$}}

\newcommand{\sba} 	{\mbox{${S/B}$}}
\newcommand{\rfw}   {\mbox{$R_2$}}
\newcommand{\effmc}     {\mbox{${\epsilon_{\rm MC}}$}}
\newcommand{\effdata}     {\mbox{${\epsilon_{\rm DATA}}$}}

\newcommand{\RR}	{\mbox{${\cal R}$}}

\newcommand{\subs}[1]{{\mbox{\scriptsize #1}}}
\newcommand{\re}{\mathrm{Re\:}}

%-----------------------------------------------------------------------
%-----------------------------------------------------------------------
%-----------------------------------------------------------------------

%\preprint line(s) will be ignored for PRL/PRD
%\preprint{CLEO Draft YY-NNA} % For paper draft CBX YY-NN -> Draft YY-NNA
%\preprint{CLEO CONF YY-NN}   % For conference papers
%\preprint{ICHEP ABSnnn}      % For conference papers
\preprint{CLNS 05/XXXX; CLEO 05-XX}       % for CLNS notes
\preprint{EPS-XXX}         % for CLNS notes

\title{Di-Lepton Widths of Upsilon(1S,2S,3S)~}

\thanks{Archived as hep-ex/XXXXXXX; 
submitted to {\it Phys. Rev. Q}}

%-------- INSERT HERE ------------
% Your author list goes here  REMOVE EVERYTHING to END INSERT and
% replace with your authorlist (ask cleoac).


\author{A.~Bornheim}
\author{E.~Lipeles}
\author{S.~P.~Pappas}
\author{A.~Shapiro}
\author{W.~M.~Sun}
\author{A.~J.~Weinstein}
\affiliation{California Institute of Technology, Pasadena, California 91125}
\author{R.~A.~Briere}
\author{G.~P.~Chen}
\author{T.~Ferguson}
\author{G.~Tatishvili}
\author{H.~Vogel}
\affiliation{Carnegie Mellon University, Pittsburgh, Pennsylvania 15213}
\author{N.~E.~Adam}
\author{J.~P.~Alexander}
\author{K.~Berkelman}
\author{F.~Blanc}
\author{V.~Boisvert}
\author{D.~G.~Cassel}
\author{P.~S.~Drell}
\author{J.~E.~Duboscq}
\author{K.~M.~Ecklund}
\author{R.~Ehrlich}
\author{R.~S.~Galik}
\author{L.~Gibbons}
\author{B.~Gittelman}
\author{S.~W.~Gray}
\author{D.~L.~Hartill}
\author{B.~K.~Heltsley}
\author{L.~Hsu}
\author{C.~D.~Jones}
\author{J.~Kandaswamy}
\author{D.~L.~Kreinick}
\author{A.~Magerkurth}
\author{H.~Mahlke-Kr\"uger}
\author{T.~O.~Meyer}
\author{N.~B.~Mistry}
\author{J.~R.~Patterson}
\author{D.~Peterson}
\author{J.~Pivarski}
\author{S.~J.~Richichi}
\author{D.~Riley}
\author{A.~J.~Sadoff}
\author{H.~Schwarthoff}
\author{M.~R.~Shepherd}
\author{J.~G.~Thayer}
\author{D.~Urner}
\author{T.~Wilksen}
\author{A.~Warburton}
\author{M.~Weinberger}
\affiliation{Cornell University, Ithaca, New York 14853}
\author{S.~B.~Athar}
\author{P.~Avery}
\author{L.~Breva-Newell}
\author{V.~Potlia}
\author{H.~Stoeck}
\author{J.~Yelton}
\affiliation{University of Florida, Gainesville, Florida 32611}
\author{K.~Benslama}
\author{B.~I.~Eisenstein}
\author{G.~D.~Gollin}
\author{I.~Karliner}
\author{N.~Lowrey}
\author{C.~Plager}
\author{C.~Sedlack}
\author{M.~Selen}
\author{J.~J.~Thaler}
\author{J.~Williams}
\affiliation{University of Illinois, Urbana-Champaign, Illinois 61801}
\author{K.~W.~Edwards}
\affiliation{Carleton University, Ottawa, Ontario, Canada K1S 5B6 \\
and the Institute of Particle Physics, Canada M5S 1A7}
\author{D.~Besson}
\author{X.~Zhao}
\affiliation{University of Kansas, Lawrence, Kansas 66045}
\author{S.~Anderson}
\author{V.~V.~Frolov}
\author{D.~T.~Gong}
\author{Y.~Kubota}
\author{S.~Z.~Li}
\author{R.~Poling}
\author{A.~Smith}
\author{C.~J.~Stepaniak}
\author{J.~Urheim}
\affiliation{University of Minnesota, Minneapolis, Minnesota 55455}
\author{Z.~Metreveli}
\author{K.K.~Seth}
\author{A.~Tomaradze}
\author{P.~Zweber}
\affiliation{Northwestern University, Evanston, Illinois 60208}
\author{S.~Ahmed}
\author{M.~S.~Alam}
\author{J.~Ernst}
\author{L.~Jian}
\author{M.~Saleem}
\author{F.~Wappler}
\affiliation{State University of New York at Albany, Albany, New York 12222}
\author{K.~Arms}
\author{E.~Eckhart}
\author{K.~K.~Gan}
\author{C.~Gwon}
\author{K.~Honscheid}
\author{D.~Hufnagel}
\author{H.~Kagan}
\author{R.~Kass}
\author{T.~K.~Pedlar}
\author{E.~von~Toerne}
\author{M.~M.~Zoeller}
\affiliation{Ohio State University, Columbus, Ohio 43210}
\author{H.~Severini}
\author{P.~Skubic}
\affiliation{University of Oklahoma, Norman, Oklahoma 73019}
\author{S.A.~Dytman}
\author{J.A.~Mueller}
\author{S.~Nam}
\author{V.~Savinov}
\affiliation{University of Pittsburgh, Pittsburgh, Pennsylvania 15260}
\author{J.~W.~Hinson}
\author{J.~Lee}
\author{D.~H.~Miller}
\author{V.~Pavlunin}
\author{B.~Sanghi}
\author{E.~I.~Shibata}
\author{I.~P.~J.~Shipsey}
\affiliation{Purdue University, West Lafayette, Indiana 47907}
\author{D.~Cronin-Hennessy}
\author{A.L.~Lyon}
\author{C.~S.~Park}
\author{W.~Park}
\author{J.~B.~Thayer}
\author{E.~H.~Thorndike}
\affiliation{University of Rochester, Rochester, New York 14627}
\author{T.~E.~Coan}
\author{Y.~S.~Gao}
\author{F.~Liu}
\author{Y.~Maravin}
\author{R.~Stroynowski}
\affiliation{Southern Methodist University, Dallas, Texas 75275}
\author{M.~Artuso}
\author{C.~Boulahouache}
\author{S.~Blusk}
\author{K.~Bukin}
\author{E.~Dambasuren}
\author{R.~Mountain}
\author{H.~Muramatsu}
\author{R.~Nandakumar}
\author{T.~Skwarnicki}
\author{S.~Stone}
\author{J.C.~Wang}
\affiliation{Syracuse University, Syracuse, New York 13244}
\author{A.~H.~Mahmood}
\affiliation{University of Texas - Pan American, Edinburg, Texas 78539}
\author{S.~E.~Csorna}
\author{I.~Danko}
\affiliation{Vanderbilt University, Nashville, Tennessee 37235}
\author{G.~Bonvicini}
\author{D.~Cinabro}
\author{M.~Dubrovin}
\author{S.~McGee}
\affiliation{Wayne State University, Detroit, Michigan 48202}
%\author{(CLEO Collaboration)} %FOR PRD_SPECIAL_CHANGEME
\collaboration{CLEO Collaboration} %FOR PRL,CLNS
\noaffiliation

%\author{John Doe}
%\affiliation{Physics Department, Cornell University
%Ithaca, New York 14853}
%\author{(CLEO Collaboration)}     %FOR PRD_SPECIAL_CHANGEME
%\collaboration{CLEO Collaboration} %FOR PRL and CLNS (superscriptaddress)
%\noaffiliation

%-------- END INSERT ------------

%please hard code the date when you have a final draft and submit to CLEOAC
\date{\today}



%---------------------------------------------------------------------
%
%\abstract
%
%---------------------------------------------------------------------

\begin{abstract} 
Abstracty-type stuff

\end{abstract}
\pacs{13.20.He}
\maketitle

%---------------------------------------------------------------------
%
\section{Introduction}
%
%---------------------------------------------------------------------

%---------------------------------------------------------------------
%
\subsection{The Meaning of $\Gamma_{ee}$ and the Motivation for a
High-Precision Measurement}
%
%---------------------------------------------------------------------

$\Gamma_{ee}$ and why it matters: measures strength of $b\bar{b}$
binding, predicted by lattice, related to $f_B$

%---------------------------------------------------------------------
%
\subsection{Outline of the Experimental Method}
%
%---------------------------------------------------------------------

experimental technique: measure coupling to incident $e^+e^-$ rather
than final state, related to total energy-integrated cross-section of
$\Upsilon$.  Resonance Breit-Wigner is convoluted with beam-energy
spread and radiative corrections: the former is Gaussian and preserves
area, but the latter diverges.  Technique: measure shape of
cross-section versus energy and fit it to a three-way convolution of
Breit-Wigner, Gaussian, and radiative corrections, with the area of
the Breit-Wigner as a fit parameter.

%---------------------------------------------------------------------
%
\section{Detectors and Datasets}
%
%---------------------------------------------------------------------

%---------------------------------------------------------------------
%
\subsection{Relevant Parts of the CLEO Detector}
%
%---------------------------------------------------------------------

Description of the DR, CC

%---------------------------------------------------------------------
%
\subsection{The Beam Energy Measurement}
%
%---------------------------------------------------------------------

Beam-energy measurement: NMR probe in a test magnet in series with
CESR dipoles; may be displaced during weekly machine studies, so beam
energy calibration isn't repeatable from one week to the next

%---------------------------------------------------------------------
%
\subsection{Lineshape Datasets}
%
%---------------------------------------------------------------------

Scan method: every week, do a complete and independent scan of an
$\Upsilon$ resonance, about 10 scans per resonance.  Continuum data
(20 MeV below resonance) and radiative corrections measurements
(20--50 MeV above resonance) is shared by all scans of a particular
resonance, insensitive to the observed $\lesssim$1 MeV shifts in beam
energy calibration.  0.1 fb$^{-1}$ continuum data, 0.1 fb$^{-1}$ scan and
radiative correction data, with 0.x fb$^{-1}$ on-resonance data that
could be added as a scan point, taken in November 2001 -- September 2002.

Tabulate the scans

%---------------------------------------------------------------------
%
\section{Event Selection and Backgrounds}
%
%---------------------------------------------------------------------

%---------------------------------------------------------------------
%
\subsection{Selection Criteria for Hadronic Events}
%
%---------------------------------------------------------------------

Continuum processes are subtracted with a parameter that floats in the
fit, pinned down by the high statistics in the continuum sample.  But
lepton pairs should be rejected to avoid large continuum subtractions
from Bhabhas: extrapolate from hadronic cross-section to total
cross-section with a factor of $(1-3\mathcal{B}_{\mu\mu})$.  Describe
cuts for hadronic events

%---------------------------------------------------------------------
%
\subsection{Selection Criteria for Luminosity Measurement}
%
%---------------------------------------------------------------------

As a measure of luminosity, we avoid $e^+e^-$ and $\mu^+\mu^-$ because
this is a final state of $\Upsilon$, and $e^+e^- \to \ell^+\ell^-$ and
$e^+e^- \to \Upsilon \to \ell^+\ell^-$ interfere destructively below
the resonance and constructively above it: one would need to be very
careful to account for interference correctly.  Instead, we use
$e^+e^- \to \gamma\gamma$ (``Gamgam'') as a measure of luminosity from
one scan point to the next: it will be calibrated (normalizing all
cross-section measurements simultaneously) with a more sophisticated
luminosity study which combines Bhabha, Gamgam, and $\mu^+\mu^-$
(``Mupair'') measurements.

Describe Gamgam cuts, explaining exclusion regions; bhabhas to measure
trigger efficiency after cuts: it's stable.  Searched for $\Upsilon
\to \gamma \chi_b \to \gamma \gamma \gamma$ as a background that would
vary through the resonance, found nothing.  [Find a way to be
quantitative or show Figure 10.1?]

%---------------------------------------------------------------------
%
\subsection{Explicitly-Subtracted Backgrounds}
%
%---------------------------------------------------------------------

Backgrounds that are proportional to luminosity will be subtracted in
the fit, but beam-gas (describe) and cosmic rays are exceptions.  Cuts
to identify and beam-gas and cosmic rays in each scan point,
single-beam and no-beam samples to measure fraction of beam-gas and
cosmic rays that survive hadron cuts.  [show Figures 5.5 without
circles, 5.6] Surviving cosmic ray count is subtracted from every scan
point as a correction ($\pm$ statistics), and 50\% $\pm$ 50\% of
surviving beam-gas of each species is subtracted from every scan
point, because of backgrounds in the beam-gas count

%---------------------------------------------------------------------
%
\section{Stability of the Cross-Section Measurement}
%
%---------------------------------------------------------------------

%---------------------------------------------------------------------
%
\subsection{Test of Drift Chamber/Calorimeter Synchronization}
%
%---------------------------------------------------------------------

Hadronic event count relies primarily on the DR, Gamgam count relies
entirely on CC, can measure the wrong cross-section if DR looses
sensitivity (HV trip) while CC continues to take data.  [Is there a
good reason the converse won't happen?]  Method: search for excesses
of trackless bhabhas, 25 examples found, all but one at the beginning
or end of the run.  [How to define a ``run''???]  10 were crucial to
lineshape scans, so the 99\% of the run without this error was used
for cross-section calculation.

%---------------------------------------------------------------------
%
\subsection{Test of Cross-Section Stability}
%
%---------------------------------------------------------------------

Cross-section during each scan point should be constant: fit a line to
cross-section versus time within each and find only statistical
deviations (and two outliers, which were dropped).

%---------------------------------------------------------------------
%
\section{Fits to the Lineshape Scans}
%
%---------------------------------------------------------------------

%---------------------------------------------------------------------
%
\subsection{Fits to Backgrounds Only}
%
%---------------------------------------------------------------------

Energy dependence of continuum backgrounds is largely $1/s$, but can
be modified three ways: the continuum sample may have a significant
two-photon fusion component, which scales as $\log s$, $R$ may vary
significantly through the $\Upsilon$ spectrum, and/or the Hadronic cut
efficiency may vary significantly through the $\Upsilon$ spectrum.
With three data points (the continuum below $\Upsilon(1S)$,
$\Upsilon(2S)$, and $\Upsilon(3S)$), we can't resolve which of these
effects are primarily responsible for deviations from $1/s$.  However,
we only need to parameterize the energy dependence of these background
processes, and as they are all approximately linear, they can be
fitted to a single parameter.  We will first fit the three continuum
points to a background function that includes a $1/s$ term and a $\log
s$ term, as though the deviation were entirely due to a two-photon
background, and then fix this effective ``two-photon fraction'' in the
lineshape fits.

Backgrounds are fit to the following function:
\begin{multline}
  \label{backgroundfit}
  b(E) = b_0 \bigg( (1 - f_{9000}) \frac{\mbox{9000 MeV}}{E^2} +
  f_{9000} \log\left(\frac{E^2}{\mbox{9000 MeV}}\right) \\
  + A_{1S} \frac{\theta(E - M_{\Upsilon(1S)})}{E - M_{\Upsilon(1S)}} +
  A_{2S} \frac{\theta(E - M_{\Upsilon(2S)})}{E - M_{\Upsilon(2S)}} \bigg)
\end{multline}
where $b_0$ sets the scale, $f_{9000}$ is the effective two-photon
fraction at 9000 MeV, and $A_{1S}$ and $A_{2S}$ represent
contributions to of $\Upsilon(1S)$ and $\Upsilon(2S)$ to
$\Upsilon(2S)$ and $\Upsilon(3S)$ backgrounds, through initial state
radiation.  (The parameters $A_{1S}$ and $A_{2S}$ were determined with
preliminary lineshape fits, and the lineshape fits were iterated with
the background fit.  Similar tails from $J/\psi$ and $\psi'$ are a
factor of two smaller and much flatter: they contribute to
$f_{9000}$.)

Equation \ref{backgroundfit} is fitted to the three continuum points,
whose luminosity has been measured with Bhabhas, rather than Gamgams,
for extra statistical precision.  Only $b_0$ and $f_{9000}$ are
allowed to float.  The scale parameter $b_0$ is ignored, as this will
be a floating parameter in the lineshape fits, and it has no meaning
without calibrating luminosity.  The effective two-photon fraction
$f_{9000}$ is 7.9 $\pm$ 0.5.  Variation of this parameter will be
shown to have a 0.003\% effect on fits for $\Gamma_{ee}$.

[Show backgrounds fit Figure]

%---------------------------------------------------------------------
%
\subsection{Fits to Each Resonance Lineshape}
%
%---------------------------------------------------------------------

Each of the three resonances were fitted independently, but within
each resonance, all scan data were combined.  Since the beam energy
calibration can vary from one scan to the next, the centroid of the
fit function was fixed at the PDG $\Upsilon$ mass and the beam energy
calibration of each scan was included as a parameter in the fit.  The
$\Upsilon(1S)$ was scanned 12 times, so the $\Upsilon(1S)$ fit has 15
free parameters, including $\alpha\Gamma_{ee}$, the beam energy spread
$\Delta E$, and the continuum background $b_0$.  The $\Upsilon(2S)$
was scanned 6 times, and therefore has 9 free parameters, and the
$\Upsilon(3S)$ was scanned 7 times, giving it 10 free parameters.

The fit function is a convolution of a Breit-Wigner of area
$\alpha\Gamma_{ee}$ and full width 53 $\pm$ 1.5 keV, 43 $\pm$ 6 keV,
26.3 $\pm$ 3.4 keV (for the $\Upsilon(1S)$, $\Upsilon(2S)$, and
$\Upsilon(2S)$, respectively), a beam energy spread Gaussian of unit
area and width $\Delta E$, and an initial-state radiation tail which
is determined by the Breit-Wigner area.  The full width is not allowed
to float in the fit, though variation by its uncertainty has at most a
0.03\% effect on $\Gamma_{ee}$.

The fit function also includes interference between the Breit-Wigner
and the continuum background.  If the resonance and the background
have amplitudes $\mathcal{A}_{BW}(E)$ and $\mathcal{A}_C(E)$,
respectively, the total cross-section is
\begin{equation}
  \sigma_\subs{tot}(E) = |\mathcal{A}_{BW}(E) + \mathcal{A}_C(E)|^2 =
  \sigma_{BW}(E) + \sigma_C(E) + 2
  \re\{\mathcal{A}_{BW}(E)^* \mathcal{A}_C(E)\}
\end{equation}
where $\sigma_{BW}(E)$ and $\sigma_C(E)$ are the Breit-Wigner and
continuum cross-sections, respectively.  Assuming a $1/s$ continuum
background, the remaining term can be parameterized as
\begin{equation}
  2 \re\{\mathcal{A}_{BW}(E)^* \mathcal{A}_C(E)\} =
  y_\subs{int} \sigma_{BW}(E) \frac{2}{\Gamma} (E - M_{\Upsilon})
\end{equation}
where $\Gamma$ is the $\Upsilon$ full width, $M_{\Upsilon}$ is the
mass of the $\Upsilon$, and $y_{int, q\bar{q}} = (2/3)\alpha R/(1-3 B_{\mu\mu})$
for interference with $q\bar{q}$ and $y_{int, \tau^+\tau^-} =
(2/3)\alpha/B_{\tau\tau}$ for interference with $\tau^+\tau^-$
\cite{kb}.  The values of $y_{int, q\bar{q}}$ are 0.018 $\pm$ 0.0006
for all three resonances, and $y_{int, \tau^+\tau^-}$ is 0.20, 0.37,
and 0.27 for the three resonances, with a 6\% uncertainty for the
$\Upsilon(1S)$ and a 100\% uncertainty for the $\Upsilon(2S)$ and
$\Upsilon(3S)$.  (This is because $\mathcal{B}_{\tau\tau}$ is much
better known for $\Upsilon(1S)$ than the other resonances.)  Variation
of these interference terms will be shown to have at most a 0.2\%
effect on fits for $\Gamma_{ee}$.

The background term for the resonance fit functions is Equation
\ref{backgroundfit}, where only $b_0$ is allowed to float in the fit.
First the $\Upsilon(1S)$ is fitted, and its area is used to set
$A_{1S}$ for the $\Upsilon(2S)$ and $\Upsilon(3S)$ fits, then the
$\Upsilon(2S)$ is fitted, and its are is used to set $A_{2S}$ for the
$\Upsilon(3S)$ fit.

Fits to the $\Upsilon$ lineshapes are presented in Figures X, Y, and
Z, with pull distributions (pull is the difference between a scan
point and the fit value, divided by the scan point uncertainty) in
Figures A, B, and C.  All fit results and systematics studies are
presented in Table $\zeta$.  The $\Upsilon(1S)$ reduced $\chi^2$ is
1.19, which has a 0.4\% confidence level.  This will be addressed in
the following Subsection.

%---------------------------------------------------------------------
%
\subsection{Upper Limit on Energy Miscalibrations During Each Scan}
%
%---------------------------------------------------------------------

The beam energy calculation does not quote an uncertainty, so we
couldn't set horizontal error bars on the scan points.  However, we
can get a rough upper limit on the uncertainty in beam energy by
adding random miscalibrations to the beam energy measurements and
observing how this affects the fit $\chi^2$.

If the beam energy calibration changes from one scan point to the
next, it does so by way of a random walk: a shift in calibration while
measuring one scan point one affects all subsequent scan points,
unless the beam energy calibration randomly fluctuates back.
Therefore, the sequence of measurements matters a great deal.  To
simulate additional miscalibration, we randomly chose miscalibrations
with a given RMS before every measurement and applied them
cumulatively through each scan, then re-fit each resonance.  This
process was repeated 100 times, and the resulting values for
$\alpha\Gamma_{ee}$ and $\chi^2$ were averaged for successful fits
(about 80--90\% of the fits were successful).

blah\ldots blah\ldots blah\ldots



%---------------------------------------------------------------------
%
\section{Hadronic Efficiency}
%
%---------------------------------------------------------------------

Overview: how we will measure efficiency

%---------------------------------------------------------------------
%
\subsection{Minimally-biased $\Upsilon(1S)$ from Di-Pion Cascades}
%
%---------------------------------------------------------------------

Constraints on the two pions, including the exclusion of curlers and
the requirement that they miss the calorimeter

Randomly selecting a two-pion hypothesis (within window) to make the
combinatoric background flat, plot the recoil mass and define signal,
sideband.  Select ``Hadron'' trigger line for this study: two tracks
are the pions, $\Upsilon(1S)$ must generate one track and one 150 MeV
shower (which is less restrictive than the chosen trigger lines, a
necessary condition for the trigger).  Can also 


\ldots


This study was repeated requiring the ``TwoTrack'' trigger line,
rather than ``Hadron,'' which makes no requirements of the
$\Upsilon(1S)$ at all.  (-3.2 $\pm$ 1.4)\% had no tracks.

%---------------------------------------------------------------------
%
\subsection{Trigger Efficiency}
%
%---------------------------------------------------------------------

%---------------------------------------------------------------------
%
\subsection{Constructing a Hadronic Efficiency Measurement}
%
%---------------------------------------------------------------------


%---------------------------------------------------------------------
%
\section{Luminosity Calibration}
%
%---------------------------------------------------------------------

%---------------------------------------------------------------------
%
\section{Conclusions}
%
%---------------------------------------------------------------------

%---------------------------------------------------------------------
%
\section{Acknowledgements}
%
%---------------------------------------------------------------------
We gratefully acknowledge the effort of the CESR staff in providing us
with excellent luminosity and running conditions.  This work was
supported by the National Science Foundation, and the U.S. Department
of Energy.

%---------------------------------------------------------------------
%
%\section{Figures}
%
%---------------------------------------------------------------------

%---------------------------------------------------------------------
%
%\section{Bibliography}
%
%---------------------------------------------------------------------
\def\endpoint{;~~}
\def\Journal#1&#2&#3(#4){#1{\bf #2}, #3 (#4)}
\def\NIM{Nucl. Instr. and Meth. }
\def\NIMA{Nucl. Instr. and Meth. A }
\def\NPB{Nucl. Phys. B }
\def\PLB{Phys. Lett. B }
\def\PRL{Phys. Rev. Lett. }
\def\PRD{Phys. Rev. D }
\newpage
\begin{thebibliography}{99}

\bibitem{cleo3det}     CLEO Collaboration, CLNS-94-1277; D.\ Peterson \etal, {\Journal\NIMA&478&142(2002)}

\bibitem{kb}     Primer on Onium Widths, but a copy with interference; K.\ Berkelman, ???

\end{thebibliography}

\end{document}




%% \begin{table}
%% \begin{center}
%% \caption{Features of Set A and Set B.}
%% \smallskip
%% \begin{tabular}{lcc}
%% \hline
%% Quantity & ~~~~Set A & ~~~~Set B\cr 
%% \hline\hline
%% Fraction of total \nbb\     & 55\%    & 45\%  \cr
%% Track Resolution            &         &       \cr
%% \hfil{`A' Coefficient}      & 0.0055  & 0.0044\cr
%% \hfil{`B' Coefficient ($\gev^{-1}$)}      & 0.0011  & 0.0010\cr
%% $K^+\pi^-$ Mode             &         &       \cr
%% \hfil{$\sigma_{\mb}$ (\mev)}  & 2.7     & 2.7   \cr
%% \hfil{$\sigma_{\de}$(\mev)}   & 22      & 19    \cr
%% \hfil{Efficiency}           & 38\%    & 45\%  \cr
%% $K^+\pi^0$ Mode             &         &       \cr
%% \hfil{$\sigma_{\mb}$ (\mev)}  & 3.1     & 3.1   \cr
%% \hfil{$\sigma_{\de}$(\mev)}   & 31      & 31    \cr
%% \hfil{Efficiency}           & 33\%    & 35\%  \cr
%% $\pi^0\pi^0$ Mode\footnote{Resolutions are given as average of low-side and high-side half-resolutions.}           &         &       \cr
%% \hfil{$\sigma_{\mb}$ (\mev)}  & 3.6     & 3.6  \cr
%% \hfil{$\sigma_{\de}$(\mev)}   & 43      & 43    \cr
%% \hfil{Efficiency}           & 22\%    & 22\%  \cr
%% \hline
%% \end{tabular}
%% \label{tab:oldnew}
%% \end{center}
%% \end{table}

%% \bibitem{fleischeretal} 
%% Y.-Y. Keum, H.-N. Li, and A.I. Sanda, arXiv:hep-ph/0201103; 
%% M. Neubert,                  {JHEP} {\bf 9902} (1999) 014;
%% M. Neubert and J.L. Rosner,  {\Journal\PRL&81&5076(1998)};

%% \bibitem{cleo3det}     CLEO Collaboration, CLNS-94-1277; D. Peterson \etal, {\Journal\NIMA&478&142(2002)}

%% \bibitem{pdg} Particle Data Group, {\Journal\PRD&66&010001(2002)}.\label{ref:pdg}
