% Prepared by Karl Ecklund September 2002
% Questions, Improvements, Comments to kme and CLEOAC
%
% LATEX 2e Template for CLEO Papers
% YOU MUST USE REVTEX4 and latex 2e
%
% Checklist for Paper Drafts:
% ---------------------------
% 0) Use appropriate \documetclass line as indicated below
% 1) Draft Number: Use latest CBX number, append A for the first vote
%                                (B,C,... for subsequent if any votes)
% 2) Don't use CLNS or CLEO numbers - this happens after your vote.
% 3) Title; use \\ to break title over several lines.
% 4) Abstract
% 5) For the Author list use CLEO Collaboration only
% 6) Body
% 
% Checklist for Journal Submissions:
% ----------------------------------
% 0) Use appropriate \documetclass line as indicated below
%    - For CLNS, hep-ex, preprints, conf. paper use CLNS version
%    - For PRL submission use PRL version
%    - For PRD submission use PRD version 
%      AND use \author{(CLEO Collaboration)} instead of \collaboration{CLEO}
%      SEE PRD_SPECIAL_CHANGEME in author list during step 5 below
% 1) CLEO paper number (from CLEOAC)
% 2) CLNS preprint number (from CLEOAC)
% 3) Title; use \\ to break title over several lines.
% 4) Abstract
% 5) Author list (from CLEOAC)
% 6) PACS codes
% 7) Body
% 8) Add acknowledgments
% 9) Hardcode the \date when ready to submit to journal and hep-ex.
%
% Checklist for Conference Papers:
% --------------------------------
% 0) Use appropriate \documetclass line as indicated below
%    - For conf. paper use CLNS version
% 1) CLEO conference paper number (from CLEOAC) (don't use CLNS)
% 2) Title; use \\ to break title over several lines.
% 3) To \thanks after title, add appropriate conference info.
% 4) Abstract
% 5) Author list (from CLEOAC)
% 6) Body
% 7) Add acknowledgments
% 8) Hardcode the \date when ready to submit to conference and hep-ex.
% 
% KNOWN PROBLEMS with template and REVTEX4
% - can't have \collaboration in PRD style grouped author list
%   Using \author{(CLEO Collaboration)} instead
% - can't make abstract appear before full author list ala CLNS notes
%   Abandoning this as the format for CLNS.

%%%%%%%%%%%%%%%%%%%%%%%%%%%%%%%%%%%%%%%%%%%%%%%%%%%%%%%%%%%%%%%%%%%%%%
% Select one of the \documentclass lines below for your paper
%%%%%%%%%%%%%%%%%%%%%%%%%%%%%%%%%%%%%%%%%%%%%%%%%%%%%%%%%%%%%%%%%%%%%%
%%%%%%%%%%%%%% Use for CLNS preprint (hep-ex) and Paper Drafts
\documentclass[aps,prd,preprint,superscriptaddress,tightenlines,nofootinbib,floatfix]{revtex4}

%%%%%%%%%%%%%% Use for PRL
%\documentclass[aps,prl,twocolumn,superscriptaddress,showpacs]{revtex4}

%%%%%%%%%%%%%% Use for PRD submission
%\documentclass[aps,prd,preprint,nopreprintnumbers,nofootinbib,showpacs]{revtex4}
%%%%%%%%%%%%%% Use for PRD formatting tables and figures in 2 column
%\documentclass[aps,prd,twocolumn,nofootinbib,showpacs]{revtex4}

\usepackage{graphicx}% Include figure files
\usepackage{dcolumn}% Align table columns on decimal point
\usepackage{bm}% bold math
\usepackage{multirow}% multirow table entries
\usepackage{amsmath}

\begin{document}

%-Definitions-----------------------------------------------------------
\newcommand{\barb}{\bar{B}}
%\newcommand{\bbbar}{B\bar{B}}
%\newcommand{\bm}{B^-}
\newcommand{\bob}{\bar{B}^0}
\newcommand{\bo}{B^0}
\newcommand{\bp}{B^+}
\newcommand{\dkpipiz}{D^0 \to K^-\pi^+\pi^0}
\newcommand{\dkpi}{D^0 \to K^-\pi^+}
\newcommand{\dkrho}{D^0 \to K^- \rho^+}
\newcommand{\dobar}{\overline{D}^0}
\newcommand{\etal}{{\it et al.}}
\newcommand{\F}{$\cal F$}
%\newcommand{\gev}{\ \rm GeV}
\newcommand{\kpipiz}{K^-\pi^+\pi^0}
%\newcommand{\kpi}{K^-\pi^+}
\newcommand{\krho}{K^- \rho^+}
%\newcommand{\mev}{\ \rm MeV}
\newcommand{\piz}{\pi^0}
\newcommand{\ra}{{\rightarrow}}
\newcommand{\Ufs}{\Upsilon(4S)}

\newcommand{\about}		{\mbox{$\sim$}}
\newcommand{\aerr}[3]   {\mbox{${{#1}^{+ #2}_{- #3}}$}}
\newcommand{\amp}    	{\mbox{${\cal A}$}}
\newcommand{\avcb}		{\mbox{$|V_{cb}|$}}
\newcommand{\avtb}		{\mbox{$|V_{tb}|$}}
\newcommand{\avtd}		{\mbox{$|V_{td}|$}}
\newcommand{\avts}		{\mbox{$|V_{ts}|$}}
\newcommand{\avub}		{\mbox{$|V_{ub}|$}}
\newcommand{\avud}		{\mbox{$|V_{ud}|$}}
\newcommand{\avus}		{\mbox{$|V_{us}|$}}
\newcommand{\BBbar}     {\mbox{$B\bar B$}}
\newcommand{\bbbar}     {\mbox{$B\bar B$}}
\newcommand{\bc}		{\mbox{$b\to c$}}
\newcommand{\bdb}		{\mbox{$\bar B^0_d $}}
\newcommand{\bd}		{\mbox{$B^0_d $}}
\newcommand{\berr}[2]   {\mbox{${{}^{+ #1}_{- #2}}$}}
\newcommand{\bmeson}	{\mbox{$B$}}
\newcommand{\bqb}		{\mbox{$\bar B^0_q $}}
\newcommand{\bq}		{\mbox{$B^0_q $}}
\newcommand{\branch}    {\mbox{${\cal B}$}}
\newcommand{\bsb}		{\mbox{$\bar B^0_s$}}
\newcommand{\bs}		{\mbox{$B^0_s$}}
\newcommand{\bu}		{\mbox{$b\to u$}}
\newcommand{\cerr}[4]   {\mbox{${{}^{+ #1}_{- #2}{}^{+ #3}_{- #4}}$}}
\newcommand{\chisq}		{\mbox{$\chi^2$}}
\newcommand{\cosB}		{\mbox{${\cos\theta_{\rm B}}$}}
\newcommand{\cossph}	{\mbox{$\cos \theta_{\rm sph}$}}
\newcommand{\costhr}	{\mbox{$\cos \theta_{\rm thr}$}}
\newcommand{\dedx}		{\mbox{$dE/dx$}}
\newcommand{\derr}[5]   {\mbox{${{#1}^{+ #2}_{- #3}{}^{+ #4}_{- #5}}$}}
\newcommand{\de}		{\mbox{$\Delta E$}}
\newcommand{\dzerobar}	{\mbox{$\overline {D^0}$}}
\newcommand{\dzero}		{\mbox{${D^0}$}}
\newcommand{\ebeam}		{\mbox{$E_{\rm beam}$}}
\newcommand{\eb}		{\mbox{$E_b$}}
\newcommand{\eeqq}		{\mbox{$e^+e^-\to\qqb$}}
\newcommand{\ee}		{\mbox{$e^+e^-$}}
%\newcommand{\etal}		{\mbox{${\it et al}$}}
\newcommand{\expt}		{\mbox{$_{\rm expt}$}}
\newcommand{\fbinv}		{\mbox{${\rm fb}^{-1}$}}
%\newcommand{\fisher}    {\mbox{${\cal F}$}}
\newcommand{\fisher}    {\mbox{$x_{\cal F}$}}
\newcommand{\gev}		{\mbox{${\rm ~GeV}$}}
\newcommand{\hh}		{\mbox{$h^+h^-$}}
\newcommand{\hpm}		{\mbox{$h^\pm$}}
%\newcommand{\implies}	{\mbox{${\Longrightarrow}$}}
\newcommand{\jimexp}[1]	{\mbox{${\rm e}^{{#1}}$}}
\newcommand{\kk}		{\mbox{KK}}
\newcommand{\Kpi}		{\mbox{$K\pi$}}
\newcommand{\kpi}		{\mbox{$\Kpi$}}
\newcommand{\kpz}		{\mbox{$K^+\pi^0$}}
\newcommand{\KP}		{\mbox{$K\pi$}}
\newcommand{\ksp}		{\mbox{$K^0_S\pi^+$}}
\newcommand{\ks}		{\mbox{$K^0_S$}}
\newcommand{\kz}		{\mbox{$K^0$}}
\newcommand{\kzb}		{\mbox{$\overline{K^0}$}}
\newcommand{\like}    	{\mbox{${\cal L}$}}
\newcommand{\Lp}		{\mbox{$\Lambda \bar p$}}
\newcommand{\lum}    	{\mbox{${\cal L}$}}
\newcommand{\mb}		{\mbox{$M_B$}}
\newcommand{\mev}		{\mbox{${\rm MeV}$}}
\newcommand{\micron}	{\mbox{$~\mu{\rm m}$}}
\newcommand{\mm}		{\mbox{$\mu^+\mu^-$}}
\newcommand{\model}		{\mbox{$_{\rm mod}$}}
\newcommand{\nbb}		{\mbox{$N_{B\bar B}$}}
\newcommand{\nbinv}		{\mbox{${\rm nb}^{-1}$}}
\newcommand{\pb}        {\mbox{$p_B$}}
\newcommand{\pbinv}		{\mbox{${\rm pb}^{-1}$}}
\newcommand{\pdf}		{\mbox{${PDF}$}}
\newcommand{\pdfs}		{\mbox{${PDF}{\rm s}$}}
\newcommand{\pidkpi}	{\mbox{$\Delta_{K\pi}$}}
\newcommand{\pidkp}		{\mbox{$\Delta_{Kp}$}}
\newcommand{\pid}		{\mbox{$\Delta_{PID}$}}
\newcommand{\pipi}		{\mbox{$\Pipi$}}
\newcommand{\Pipi}		{\mbox{$\pi\pi$}}
\newcommand{\power}[1]  {\mbox{${\times 10^{#1}}$}}
\newcommand{\ppz}		{\mbox{$\pi^+\pi^0$}}
\newcommand{\pvec}		{\mbox{$\vec{p}$}}
\newcommand{\pz}		{\mbox{$\pi^0$}}
\newcommand{\qqb}		{\mbox{$q\bar q$}}
\newcommand{\qq}		{\mbox{${q\bar q}$}}
\newcommand{\stat}		{\mbox{$_{\rm stat}$}}
\newcommand{\syst}		{\mbox{$_{\rm syst}$}}
\newcommand{\theo}		{\mbox{$_{\rm theo}$}}
\newcommand{\upsi}		{\mbox{$\Upsilon$({\rm 4S})}}
\newcommand{\vcb}		{\mbox{$V_{cb}$}}
\newcommand{\vcd}		{\mbox{$V_{cd}$}}
\newcommand{\vcs}		{\mbox{$V_{cs}$}}
\newcommand{\vtb}		{\mbox{$V_{tb}$}}
\newcommand{\vtd}		{\mbox{$V_{td}$}}
\newcommand{\vts}		{\mbox{$V_{ts}$}}
\newcommand{\vub}		{\mbox{$V_{ub}$}}
\newcommand{\vud}		{\mbox{$V_{ud}$}}
\newcommand{\vus}		{\mbox{$V_{us}$}}
\newcommand{\zhat}		{\mbox{$\hat{\bf z}$}}

\newcommand{\ppbar}		{\mbox{${p\bar p}$}}
\newcommand{\pL}		{\mbox{${p\bar\Lambda}$}}
\newcommand{\LL}		{\mbox{${\Lambda\bar\Lambda}$}}

\newcommand{\mpdf}    	{\mbox{${\cal M}$}}
\newcommand{\epdf}    	{\mbox{${\cal E}$}}
\newcommand{\fpdf}    	{\mbox{${\cal F}$}}
\newcommand{\cpdf}    	{\mbox{${\cal C}$}}
\newcommand{\dkmode}	{\mbox{${\mu}$}}
\newcommand{\contrib}	{\mbox{${\kappa}$}}
\newcommand{\mc}    	{\mbox{${}_{\dkmode\contrib}$}}
\newcommand{\nk}    	{\mbox{${n}_{\contrib}$}}

\newcommand{\G}		{\mbox{${G}$}}
\newcommand{\GG}	{\mbox{${\cal G}$}}
\newcommand{\ARG}	{\mbox{${A}$}}
\newcommand{\LIN}	{\mbox{${L}$}}
\newcommand{\BW}	{\mbox{${\cal R}$}}
\newcommand{\FI}	{\mbox{${F_0}$}}
\newcommand{\DG}	{\mbox{${a\G_1+b\G_2}$}}
\newcommand{\GGG}	{\mbox{${a\G+b\GG}$}}
\newcommand{\DGG}	{\mbox{${a\GG_1+b\GG_2}$}}

\newcommand{\sba} 	{\mbox{${S/B}$}}
\newcommand{\rfw}   {\mbox{$R_2$}}
\newcommand{\effmc}     {\mbox{${\epsilon_{\rm MC}}$}}
\newcommand{\effdata}     {\mbox{${\epsilon_{\rm DATA}}$}}

\newcommand{\RR}	{\mbox{${\cal R}$}}

\newcommand{\subs}[1]{{\mbox{\scriptsize #1}}}
\newcommand{\re}{\mathrm{Re\:}}

%-----------------------------------------------------------------------
%-----------------------------------------------------------------------
%-----------------------------------------------------------------------

%\preprint line(s) will be ignored for PRL/PRD
%\preprint{CLEO Draft YY-NNA} % For paper draft CBX YY-NN -> Draft YY-NNA
%\preprint{CLEO CONF YY-NN}   % For conference papers
%\preprint{ICHEP ABSnnn}      % For conference papers
\preprint{CLNS 05/XXXX; CLEO 05-XX}       % for CLNS notes
\preprint{EPS-XXX}         % for CLNS notes

\title{Di-Lepton Widths of Upsilon(1S,2S,3S)~}

\thanks{Archived as hep-ex/XXXXXXX; 
submitted to {\it Phys. Rev. Q}}

%-------- INSERT HERE ------------
% Your author list goes here  REMOVE EVERYTHING to END INSERT and
% replace with your authorlist (ask cleoac).


\author{A.~Bornheim}
\author{E.~Lipeles}
\author{S.~P.~Pappas}
\author{A.~Shapiro}
\author{W.~M.~Sun}
\author{A.~J.~Weinstein}
\affiliation{California Institute of Technology, Pasadena, California 91125}
\author{R.~A.~Briere}
\author{G.~P.~Chen}
\author{T.~Ferguson}
\author{G.~Tatishvili}
\author{H.~Vogel}
\affiliation{Carnegie Mellon University, Pittsburgh, Pennsylvania 15213}
\author{N.~E.~Adam}
\author{J.~P.~Alexander}
\author{K.~Berkelman}
\author{F.~Blanc}
\author{V.~Boisvert}
\author{D.~G.~Cassel}
\author{P.~S.~Drell}
\author{J.~E.~Duboscq}
\author{K.~M.~Ecklund}
\author{R.~Ehrlich}
\author{R.~S.~Galik}
\author{L.~Gibbons}
\author{B.~Gittelman}
\author{S.~W.~Gray}
\author{D.~L.~Hartill}
\author{B.~K.~Heltsley}
\author{L.~Hsu}
\author{C.~D.~Jones}
\author{J.~Kandaswamy}
\author{D.~L.~Kreinick}
\author{A.~Magerkurth}
\author{H.~Mahlke-Kr\"uger}
\author{T.~O.~Meyer}
\author{N.~B.~Mistry}
\author{J.~R.~Patterson}
\author{D.~Peterson}
\author{J.~Pivarski}
\author{S.~J.~Richichi}
\author{D.~Riley}
\author{A.~J.~Sadoff}
\author{H.~Schwarthoff}
\author{M.~R.~Shepherd}
\author{J.~G.~Thayer}
\author{D.~Urner}
\author{T.~Wilksen}
\author{A.~Warburton}
\author{M.~Weinberger}
\affiliation{Cornell University, Ithaca, New York 14853}
\author{S.~B.~Athar}
\author{P.~Avery}
\author{L.~Breva-Newell}
\author{V.~Potlia}
\author{H.~Stoeck}
\author{J.~Yelton}
\affiliation{University of Florida, Gainesville, Florida 32611}
\author{K.~Benslama}
\author{B.~I.~Eisenstein}
\author{G.~D.~Gollin}
\author{I.~Karliner}
\author{N.~Lowrey}
\author{C.~Plager}
\author{C.~Sedlack}
\author{M.~Selen}
\author{J.~J.~Thaler}
\author{J.~Williams}
\affiliation{University of Illinois, Urbana-Champaign, Illinois 61801}
\author{K.~W.~Edwards}
\affiliation{Carleton University, Ottawa, Ontario, Canada K1S 5B6 \\
and the Institute of Particle Physics, Canada M5S 1A7}
\author{D.~Besson}
\author{X.~Zhao}
\affiliation{University of Kansas, Lawrence, Kansas 66045}
\author{S.~Anderson}
\author{V.~V.~Frolov}
\author{D.~T.~Gong}
\author{Y.~Kubota}
\author{S.~Z.~Li}
\author{R.~Poling}
\author{A.~Smith}
\author{C.~J.~Stepaniak}
\author{J.~Urheim}
\affiliation{University of Minnesota, Minneapolis, Minnesota 55455}
\author{Z.~Metreveli}
\author{K.K.~Seth}
\author{A.~Tomaradze}
\author{P.~Zweber}
\affiliation{Northwestern University, Evanston, Illinois 60208}
\author{S.~Ahmed}
\author{M.~S.~Alam}
\author{J.~Ernst}
\author{L.~Jian}
\author{M.~Saleem}
\author{F.~Wappler}
\affiliation{State University of New York at Albany, Albany, New York 12222}
\author{K.~Arms}
\author{E.~Eckhart}
\author{K.~K.~Gan}
\author{C.~Gwon}
\author{K.~Honscheid}
\author{D.~Hufnagel}
\author{H.~Kagan}
\author{R.~Kass}
\author{T.~K.~Pedlar}
\author{E.~von~Toerne}
\author{M.~M.~Zoeller}
\affiliation{Ohio State University, Columbus, Ohio 43210}
\author{H.~Severini}
\author{P.~Skubic}
\affiliation{University of Oklahoma, Norman, Oklahoma 73019}
\author{S.A.~Dytman}
\author{J.A.~Mueller}
\author{S.~Nam}
\author{V.~Savinov}
\affiliation{University of Pittsburgh, Pittsburgh, Pennsylvania 15260}
\author{J.~W.~Hinson}
\author{J.~Lee}
\author{D.~H.~Miller}
\author{V.~Pavlunin}
\author{B.~Sanghi}
\author{E.~I.~Shibata}
\author{I.~P.~J.~Shipsey}
\affiliation{Purdue University, West Lafayette, Indiana 47907}
\author{D.~Cronin-Hennessy}
\author{A.L.~Lyon}
\author{C.~S.~Park}
\author{W.~Park}
\author{J.~B.~Thayer}
\author{E.~H.~Thorndike}
\affiliation{University of Rochester, Rochester, New York 14627}
\author{T.~E.~Coan}
\author{Y.~S.~Gao}
\author{F.~Liu}
\author{Y.~Maravin}
\author{R.~Stroynowski}
\affiliation{Southern Methodist University, Dallas, Texas 75275}
\author{M.~Artuso}
\author{C.~Boulahouache}
\author{S.~Blusk}
\author{K.~Bukin}
\author{E.~Dambasuren}
\author{R.~Mountain}
\author{H.~Muramatsu}
\author{R.~Nandakumar}
\author{T.~Skwarnicki}
\author{S.~Stone}
\author{J.C.~Wang}
\affiliation{Syracuse University, Syracuse, New York 13244}
\author{A.~H.~Mahmood}
\affiliation{University of Texas - Pan American, Edinburg, Texas 78539}
\author{S.~E.~Csorna}
\author{I.~Danko}
\affiliation{Vanderbilt University, Nashville, Tennessee 37235}
\author{G.~Bonvicini}
\author{D.~Cinabro}
\author{M.~Dubrovin}
\author{S.~McGee}
\affiliation{Wayne State University, Detroit, Michigan 48202}
%\author{(CLEO Collaboration)} %FOR PRD_SPECIAL_CHANGEME
\collaboration{CLEO Collaboration} %FOR PRL,CLNS
\noaffiliation

%\author{John Doe}
%\affiliation{Physics Department, Cornell University
%Ithaca, New York 14853}
%\author{(CLEO Collaboration)}     %FOR PRD_SPECIAL_CHANGEME
%\collaboration{CLEO Collaboration} %FOR PRL and CLNS (superscriptaddress)
%\noaffiliation

%-------- END INSERT ------------

%please hard code the date when you have a final draft and submit to CLEOAC
\date{\today}



%---------------------------------------------------------------------
%
%\abstract
%
%---------------------------------------------------------------------

\begin{abstract} 
Abstracty-type stuff

\end{abstract}
\pacs{13.20.He}
\maketitle

%---------------------------------------------------------------------
%
\section{Introduction}
%
%---------------------------------------------------------------------

%---------------------------------------------------------------------
%
\subsection{The Meaning of $\bf \Gamma_{ee}$ and Motivation for a High-Precision Measurement}
%
%---------------------------------------------------------------------

%% partial width of $\Upsilon$ to $e^+e^-$, or $\Gamma_{ee} =
%% \mathcal{B}_{ee} \Gamma \ref{eqn:beegamma}$ where $\Gamma$ is the full width of
%% $\Upsilon$.

%% Related to the shape of the $b\bar{b}$ spacial wavefunction:
%% $\Gamma_{ee}$ = $\displaystyle \left(\frac{16 \pi {\alpha_{QED}}^2}{3
%% {M_\Upsilon}^2}\right) | \psi(0) |^2$ \cite{peskin}.  Therefore characterizes the
%% strength of QCD: it measures how big an $\Upsilon$ the QCD potential
%% permits

%% Most importantly, can be calculated in lattice QCD without
%% ``quenching'': i.e. can be calculated accurately.  Anticipated
%% accuracy of lattice $\Gamma_{ee}$ calculations is better than the
%% current experimental precision (on the 2S and 3S at least) \cite{davies}, so a new
%% set of high-precision measurements would test the predictive power of
%% this new technique.  Also, notice the similarity of the process
%% measured by $\Gamma_{ee}$ to that of $f_B$ (Figure \ref{fig:diagrams}).  The QCD physics
%% differs only in the mass of one quark in a central force problem:
%% validation of the new lattice techniques would lend credence to a
%% lattice calculation of $f_B$.

%---------------------------------------------------------------------
%
\subsection{Outline of the Experimental Method}
%
%---------------------------------------------------------------------

%% not measured with $\Upsilon \to e^+e^-$, but with $e^+e^- \to
%% \Upsilon$.  This is because $\Upsilon(1S,2S,3S)$ are narrower than
%% beam energy spread, so $\Gamma$ can't be measured directly to
%% calculate $\Gamma_{ee}$ through Equation
%% \ref{eqn:beegamma}. (Actually, $\Gamma$ is calculated from
%% $\Gamma_{ee}$.)  Measurement takes advantage of
%% \begin{equation}
%%   \Gamma_{ee} = \left(\frac{{M_\Upsilon}^2}{6 \pi^2}\right) \int
%%   \sigma(e^+e^- \to \Upsilon) \, dE \label{eqn:areatogammaee}
%% \end{equation} \cite{pdgupsilonintro} \cite{peskin}

%% Beam energy spread convolves the cross-section measurement, but this
%% leaves $\sigma(e^+e^- \to \Upsilon) \, dE$ unaffected.  Shape of the
%% spectral line is also convolved with a high-energy tail due to
%% incident $e^+e^-$ radiating down $M_\Upsilon$ and forming an
%% $\Upsilon$, and this correction actually diverges.  It can be
%% accommodated by constructing a fit function which is a convolution of
%% a Breit-Wigner, a Gaussian beam energy spread, and QED corrections
%% calculated by \cite{kf}, and fitting to measurements of cross-section
%% through the $\Upsilon$ lineshape.  $\Gamma_{ee}$ will be a parameter
%% in that fit.

%---------------------------------------------------------------------
%
\section{Detectors and Datasets}
%
%---------------------------------------------------------------------

%---------------------------------------------------------------------
%
\subsection{Relevant Parts of the CLEO Detector}
%
%---------------------------------------------------------------------

%% The CLEO detector, which was used for this analysis, is a nearly $4
%% \pi$ multipurpose detector situated at the collision point of the
%% Cornell Electron Storage Ring (CESR).  Only three subdetectors were
%% used: the central drift chamber and silicon vertex device for tracking,
%% and the CsI crystal calorimeter for detecting neutral particles.

%---------------------------------------------------------------------
%
\subsection{The Beam Energy Measurement}
%
%---------------------------------------------------------------------

%% Beam energy was determined by directly measuring the dipole magnetic
%% field in a test magnet.  This test magnet is identical to those used
%% to store the incident beams, and it is in series with them to have the
%% same current and therefore the same magnetic field.  The magnetic
%% field is measured with an NMR probe inside the beampipe of the test
%% magnet.  While beam energy is reproducible during operation, this test
%% magnet may be disassembed during weekly machine studies, during which
%% the calibration of the beam energy measurement might change.

%---------------------------------------------------------------------
%
\subsection{Lineshape Datasets}
%
%---------------------------------------------------------------------

%% CESR/CLEO devoted 0.1 fb$^{-1}$, 0.1 fb$^{-1}$, and 0.1 fb$^{-1}$ to
%% the scan of each of the first three $\Upsilon$ resonances, from
%% November 2001 through September 2002.  The three scans were divided
%% into weekly miniscans, each of which covered the entire shape of the
%% resonance.  Each of these miniscans is independent enough to determine
%% its own beam energy calibration.  X fb$^{-1}$, X fb$^{-1}$, and X
%% fb$^{-1}$ of general purpose data acquired at the peak of each
%% resonance was also added to the miniscans, as long as the peak and its
%% associated miniscan were not separated by a machine studies period.
%% Also, 0.1 fb$^{-1}$, 0.1 fb$^{-1}$, and 0.1 fb$^{-1}$ of off-resonance
%% data 20 MeV below each resonance and 0.x fb$^{-1}$, 0.x fb$^{-1}$, and
%% 0.x fb$^{-1}$ of high-energy tail data 20--50 MeV above the resonance
%% were also added to the three resonance scans.  (Being far from the
%% resonance peak, these are insensitive to the $\lesssim$1 MeV beam
%% energy calibration shifts.)

%% These lineshape data are presented in Figure \ref{fig:scanfits}.

%---------------------------------------------------------------------
%
\section{Event Selection and Backgrounds}
%
%---------------------------------------------------------------------

%% To suppress bhabha backgrounds, only hadronic final states of
%% $\Upsilon$ were selected.  The hadronic cross-section is converted
%% into a total cross-section by dividing by a factor of
%% $(1-3\mathcal{B}_{\mu\mu})$.

%---------------------------------------------------------------------
%
\subsection{Identifying Hadronic $\Upsilon$ Decays}
%
%---------------------------------------------------------------------

%% Cosmic rays were rejected by requiring the distance of closest
%% approach of the closest track to the beam line ($\d_{XY}$) to be less
%% than 5 mm (a very loose requirement: this distribution is about 0.2 mm
%% wide).  Beam-gas events, in which one incident beam collides with a
%% gas atom inside the beampipe, are rejected by requiring the Z of the
%% event vertex ($d_Z$, calculated from intersections of tracks near
%% the beam spot) to be less than 7.5 cm (also loose: this
%% distribution is about 1 cm wide).  Beam-wall events were observed to
%% be less common than beam-gas, and they, too, are widely distributed in
%% $d_Z$.  (We will return to this later.)

%% Electron-positron ($e^+e^-$) and muon-pair ($\mu^+\mu^-$) final states
%% are rejected by requiring that the largest track momentum
%% ($|\vec{p}_1|$) be less than 80\% of the beam energy.  (This selection
%% is loose for all hadronic final states except for $\Upsilon' \to
%% \pi\pi\Upsilon \to \pi\pi \ell^+\ell^-$ where $\ell^\pm$ is an
%% electron/positron or a muon.)  Two-photon fusion events (and other
%% low-energy backgrounds) were rejected by requiring the visible energy
%% ($E_{vis}$, the sum of track energies ($\sqrt{{m_\pi}^2 + |\vec{p}|^2}$) and unmatched shower energies)
%% to be greater than 40\% of the center-of-mass energy (this
%% is approximately 99\% efficient).

%% These four event variables are plotted in Figure \ref{fig:cuts}.

%---------------------------------------------------------------------
%
\subsection{Backgrounds in the Hadronic Event Sample}
%
%---------------------------------------------------------------------

%% Biggest background is continuum hadrons, which will be effectively
%% subtracted by a parameter in the lineshape fit.  (This is the
%% importance of the off-resonance data point.)  Continuum hadrons (as
%% well as tau-pairs) are projected under the lineshape by assuming a
%% $1/s$ dependence with beam energy.  Two-photon fusion events may also
%% survive the $E_{vis}$ requirement, and these depend on beam energy as
%% $\log s$.  A ratio of (7.9 $\pm$ 0.5)\% two-photon events to continuum
%% events was measured at a center-of-mass energy of 9000 MeV by fitting
%% the $\Upsilon(1S)$, $\Upsilon(2S)$, and $\Upsilon(3S)$ off-resonance
%% data to a curve that included $1/s$, $\log s$, and $\Upsilon$
%% high-energy tail terms.  This ratio may actually represent variation
%% of the hadronic selection efficiency for continuum events, or an
%% energy dependence of $R$, but it parameterizes the background energy
%% dependence in the 70 MeV (out of 10 GeV) from the off-resonance
%% cross-section measurement to the high-energy tail measurement.

%% All backgrounds that are proportional to integrated luminosity are
%% represented by the background parameter in the lineshape fit.  Cosmic
%% ray and beam-gas backgrounds, however, need to be subtracted
%% explicitly.  After event selection, cosmic rays typically make up
%% 0.5\% of the continuum background, electron-induced beam-gas 0.05\%,
%% and positron-induced beam-gas 0.1\%.  Cosmic rays and beam-gas are
%% counted in every data sample by selecting events with the right
%% geometries: cosmic rays generate two collinear tracks with $d_{XY}$
%% $>$ 5 mm, and beam-gas events have no collinear tracks with $d_Z$ $>$
%% 7.5 cm.  (The sum of the tracks' Z momenta points in the direction of
%% the incident particle, so that electron-induced beam-gas can be
%% distinguished from positron-induced beam-gas with approximately 95\%
%% purity, if the thresholds are set at $\pm$10\% of beam energy.)

%% A sample of electron-only data (in which no positron beam was filled
%% in CESR), positron-only data, and no-beam data were collected to
%% translate these raw numbers of backgrounds into a number of events
%% that survive hadronic selections (``cuts'').  The no-beam data can be
%% assumed to be all cosmic rays, and the electon-only and positron-only
%% contain cosmic rays and beam-gas events (as well as beam-wall).  The
%% raw number of cosmic rays in the data sample, for instance, is
%% multiplied by $($\#no-beam events surviving hadron cuts$/$\#no-beam
%% events surviving cosmic ray cuts$)$ to obtain a number of cosmic rays
%% that survive hadronic cuts in the data sample.  Beam-gas counts are
%% slightly more complicated because cosmic rays must first be subtracted
%% from the single-beam datasets.

%% Figure \ref{fig:crbgcutquality} (5.6 in big paper) compares single-
%% and no-beam beam-gas and cosmic rays with those selected from the full
%% dataset.  We see that cosmic rays are cleanly separated from other
%% events, while this unclear for beam-gas events.  Therefore, only
%% statistical errors are propagated when subtracting cosmic rays from
%% the raw number of hadrons, and 50\% $\pm$ 50\% of the beam-gas
%% correction is applied.  Beam-wall events are less common than beam-gas
%% and share some of the same scaling properties (they are both
%% proportional to incident current squared), so this correction is
%% subsumed under the large uncertainty attributed to the beam-gas
%% correction.

%% Only 43\% of the $\Upsilon \to \tau^+\tau^-$ events are rejected by
%% hadronic selections, so they are added as a separate term in the
%% fit function, including the expected interference between $e^+e^-
%% \to \Upsilon \to \tau^+\tau^-$ and $e^+e^- \to \tau^+\tau^-$.

%% All backgrounds under the $\Upsilon(3S)$ resonance are plotted in
%% Figure \ref{fig:awesome} as an illustration of their relative
%% magnitudes.

%---------------------------------------------------------------------
%
\subsection{Hadronic Efficiency and Total $\Upsilon$ Efficiency}
%
%---------------------------------------------------------------------

This is an inclusive analysis, so we must set a limit on unknown
$\Upsilon$ decay types which may not be in the Monte Carlo simulation,
which will be the largest uncertainty in the measurement of hadronic
efficiency.  This limit will be measured in the $\Upsilon(1S)$ and
will be assumed to be valid for the $\Upsilon(2S)$ and $\Upsilon(3S)$,
as it is really a limit on new physics or unknown correlations in
$ggg$, $gg\gamma$, and $q\bar{q}$ hadronization.  (These processes are
not likely to be affected by a 9\% change in center-of-mass energy,
all other parameters being equal.)  We chose to measure trigger
efficiency first, so this uncertainty is largest because it includes
this provision for unknown decays.

%---------------------------------------------------------------------
%
\subsubsection{Hadronic Trigger Studies}
%
%---------------------------------------------------------------------

The hadronic trigger used in this analysis depends on a minimum number
of tracks being found in the drift chamber and clusters of energy in
the calorimeter ((3 tracks and 1 150 MeV cluster) or (2 extended
tracks and (2 150 MeV clusters or 1 750 MeV cluster)) or (1 track and
1 750 MeV cluster)).  The Monte Carlo predicts very high trigger
efficiencies: 99.8\%, 99.6\% and 99.6\% for the three resonances
($\Upsilon(2S)$ and $\Upsilon(3S)$ lose events to
$\pi^0\pi^0\ell^+\ell^-$, where $\ell^+\ell^-$ go down the beampipe).
The trigger efficiency will be studied by comparing Monte Carlo
predictions with data, first for the calorimeter's contribution to the
trigger, and then for the drift chamber's.  The total systematic error
will be 0.68\%, about a factor of two to three larger than the effects
studied.  These large errors account for the possibility of invisible
decays of $\Upsilon$, unknown to the Monte Carlo.

To check the calorimeter simulation, we selected events with the same
$d_{XY}$, $d_Z$, and $|\vec{p}_1|$ cuts as our hadronic sample, but
additionally required two charged tracks and replaced the requirement
that the sum of charged and neutral energy ($E_{vis}$) be greater than
40\% of the center-of-mass energy with one in which the charged energy
alone must be greater than 15\% of the center-of-mass energy.  (This
rejects the same background without invoking the calorimeter.)  We
also requested that these events were accepted by a diagnostic trigger
that is independent of the calorimeter, requiring only two tracks
(though it is prescaled by a factor or 19, limiting the statistical
precision of this test).  We then measured the efficiency of the
hadronic trigger in these drift chamber-selected data (after
subtracting continuum backgrounds) and in Monte Carlo with the same
cuts, and listed these efficiencies in Table \ref{tab:cccheck}.  No
statistically-significant discrepancies are observed, so the
systematic error in the Monte Carlo's calorimeter trigger simulation
is taken to be 0.45\%, the sum of the $\Upsilon(1S)$ data/Monte Carlo
difference and its uncertainty in quadrature.

\begin{table}[t]
  \caption{\label{tab:cccheck} Efficiency of the hadronic trigger in
  drift chamber-selected data (after subtracting continuum
  backgrounds) and Monte Carlo with the same cuts.  Uncertainties in
  Monte Carlo are negligible.}
  \begin{center}
    \begin{tabular}{p{5 cm} c c c}
      \hline\hline
      & $\Upsilon(1S)$ & $\Upsilon(2S)$ & $\Upsilon(3S)$ \\\hline
      data & 99.45 $\pm$ 0.30\% & 99.01 $\pm$ 0.51\% & 97.67 $\pm$ 1.17\% \\
      Monte Carlo & 99.79\% & 99.76\% & 99.73\% \\\hline\hline
    \end{tabular}
  \end{center}
\end{table}

No low-bias all-neutral trigger exists for CLEO-III, so the tracking
trigger simulation had to be checked in a different way.  We
accomplished this by varying inputs to a toy Monte Carlo.  The toy
Monte Carlo randomly selects a number of fully reconstructed tracks
(according to an input distribution from the full Monte Carlo or from
data), and correlates this to the trigger distributions that are used
to make a trigger decision.  For example, a track multiplicity is
presented in Figure \ref{fig:toymcexamples}-a: the toy Monte Carlo may
randomly choose four tracks (with 7\% probability) or zero tracks
(0.5\%), and this affects which ``CBMD'' distribution is used to pick
a random number of 750 MeV clusters from Figure
\ref{fig:toymcexamples}-b.  This way, many of the correlations present
in real events are reproduced in the toy Monte Carlo.  All of the
trigger variables are correlated to the number of reconstructed tracks
because this number classifies event types by their topologies.  We
can then replace input distributions, such as the number of trigger
tracks (``AXIAL'') assuming four fully reconstructed tracks pictured
in Figure \ref{fig:toymcexamples}-c: the solid histogram came from the
full Monte Carlo, and the dotted histogram came from data.  By
observing the change this makes in hadronic trigger efficiency, we can
deduce its sensitivity to these distributions.

\begin{figure}[t]
  \begin{center}
    \includegraphics[width=\linewidth]{../plots/trigger_toymcexamples2}
  \end{center}
  \vspace{-1.3 cm}
  \caption{\label{fig:toymcexamples} A few sample distributions which
    are used by the Toy Monte Carlo: a.\ \#tracks from the full Monte
    Carlo, b.\ \#CBMD (750 MeV cluster) distributions from the full
    Monte Carlo, the solid histogram in the case of zero tracks, and
    the dotted line in the case of four, and c.\ \#AXIAL (trigger
    track) distributions from the full Monte Carlo (solid) and from
    data (dotted), each in the case of four fully reconstructed
    tracks.}
\end{figure}

Five tests were performed with the toy Monte Carlo (toyMC).  First,
all input distributions were derived from the full Monte Carlo
(fullMC) and the toyMC hadronic trigger efficiency was shown to agree
with the fullMC's to 0.14\%.  Second, the trigger track distributions
(``AXIAL,'' in Figure \ref{fig:toymcexamples}-c) from fullMC were
replaced with distributions from data, and with distributions from
fullMC tuned to have the same trigger tracking efficiency as data (two
ways of measuring the same sensitivity).  The trigger efficiency was
insensitive to these changes to 0.18\%.  Third, the reconstructed
tracks distribution from fullMC was swapped with a distribution from
data (from the unbiased $\Upsilon(1S)$ study described below, Figure
\ref{fig:unbiasedtracks}), changing the trigger efficiency by 0.14\%.
The fourth test measured sensitivity to unknown $\Upsilon$ decays: the
0-tracks and 1-track bins were increased by one standard deviation,
which reduced trigger efficiency by 0.43\%.  The last test validates
the use of reconstructed tracks as a measure of charged multiplicity
by varying the track-finding efficiency by its uncertainty of 2\%.
The biggest trigger efficiency change in this last test was 0.08\%,
which is close to the toyMC statistical uncertainty of 0.03\%.  Added
in quadrature, these uncertainties total 0.51\%, which will be taken
as the systematic error in the drift chamber trigger simulation.

The total systematic error in trigger efficiency, including the
possibility of unknown $\Upsilon$ decays, is $\sqrt{\mbox{0.45\%}^2 +
\mbox{0.51\%}^2}$, or 0.68\%.  This error is assumed to be valid for all three
resonances.

%---------------------------------------------------------------------
%
\subsubsection{Unbiased and Low-Bias $\Upsilon(1S)$ Studies from Di-Pion Cascades}
%
%---------------------------------------------------------------------

Hadronic efficiency of $\Upsilon(1S)$ can be studied in the decay of
$\Upsilon(2S) \to \pi^+\pi^- \Upsilon(1S)$ cascades.  Since the two
charged pions can be chosen to satisfy trigger requirements, the
resulting set of $\Upsilon(1S)$ events can be free of selection bias.
(The two pions were also chosen to have sufficient $|p_z|$ to get
out of the detector before completing an orbit in the magnetic
field, to reduce track confusion.)
Two studies were performed.  In the first, cascade events were
accepted only from the prescaled two-track trigger.  This trigger's
requirements were satisfied entirely by the two pions, so the
resulting $\Upsilon(1S)$ dataset is completely unbiased, and it has
low statistics due to the prescale.  In the second, cascade events
were accepted from a trigger that required three tracks and one 150
MeV cluster: the $\Upsilon(1S)$ must provide the third track and
possibly the 150 MeV cluster.

The $\Upsilon(1S)$ recoil mass, constructed from the two pions, is
presented in Figure \ref{fig:recoilmass}.  The combinatoric background
was subtracted by identifying a sideband region and normalizing it to
the combinatoric background in the signal region through a quadratic
fit.  (The quadratic term is small and orthogonal to the other two
terms, so it can be fluctuated by its uncertainty for a systematic
error.)

\begin{figure}[p]
  \includegraphics[width=\linewidth]{../plots/proceedings_recoilmass}
  \caption{\label{fig:recoilmass} Recoil mass of the two pions in
    $\Upsilon(2S) \to \pi^+\pi^- \Upsilon(1S)$ from the unbiased
    sample (prescaled diagnostic trigger), and from the low-bias
    sample ($\Upsilon(1S)$ must produce one track and possibly one 150
    MeV cluster).  The dotted lines separate the signal region from
    the sideband, the arrow points out the expected $\Upsilon(1S)$
    mass, and the dashed line is a quadratic fit to the combinatoric
    background.}
\end{figure}

The unbiased track multiplicity distribution needed for the toyMC
trigger efficiency study was drawn from the unbiased $\Upsilon(1S)$
dataset, after subtracting combinatoric backgrounds.  It is plotted in
Figure \ref{fig:unbiasedtracks} with Monte Carlo for comparison.  The
zero-track bin has fluctuated below zero (because of the combinatoric
background subtraction) 1.7 standard deviations (when uncertainty in
the quadratic term in the combinatoric background fit is taken into
account).  In the toyMC, $\Upsilon(1S) \to e^+e^-$, $\mu^+\mu^-$, and
$\tau^+\tau^-$ were subtracted from the distribution, and the number
of zero-track events was raised to zero with an uncertainty derived
from the 10\% of the probability distribution that extended into the
physical region.
%% (101.8 $\pm$ 1.5)\% trigger efficiency --> (100 $\pm$ 1.0)\%

\begin{figure}[p]
  \includegraphics[width=\linewidth]{../plots/proceedings_tracks}
  \caption{\label{fig:unbiasedtracks} Unbiased number of tracks
    distribution from data (crosses) and Monte Carlo (histogram).  The
    cross-hatched histogram is Monte Carlo $\Upsilon(1S) \to e^+e^-$,
    $\mu^+\mu^-$, and $\tau^+\tau^-$, which is subtracted from the
    data for hadronic charged multiplicity.}
\end{figure}

Using the low-bias dataset, $\Upsilon(1S)$ events were passed through
the hadronic event selection criteria to determine its efficiency.
The bias imposed by selecting $\Upsilon(1S)$ events with one track and
one 150 MeV cluster is strictly less than that imposed by the trigger,
whose efficiency is 99.8 $\pm$ 0.68\%.  The four variables that define
the hadronic selection are plotted in Figure \ref{fig:cuts}.

\begin{figure}[p]
  \includegraphics[width=\linewidth]{../plots/proceedings_cuts2}
  \caption{\label{fig:cuts} The four variables used to select hadronic
    events, as seen in low-bias $\Upsilon(2S) \to \pi^+\pi^-
    \Upsilon(1S)$.  Dotted lines indicate cut thresholds, crosses are
    data, and histograms are Monte Carlo.  The hatched histogram in
    $|\vec{p}_1|$ represents $e^+e^-$ and $\mu^+\mu^-$ final states.}
\end{figure}

The total measured $\Upsilon(1S)$ efficiency is (92.58 $\pm$ 0.13)\%.
Most of this inefficiency is due to the leptonic modes: $\Upsilon(1S)
\to e^+e^-$ and $\mu^+\mu^-$ efficiencies are 0.6\% and 0.3\%,
respectively, and $\Upsilon(1S) \to \tau^+\tau^-$ has a 62\%
efficiency.  Correcting for leptonic modes (using $\mathcal{B}_{ee}$ =
(2.38 $\pm$ 0.11)\%, $\mathcal{B}_{\mu\mu}$ = (2.49 $\pm$ 0.07)\%, and
$\mathcal{B}_{\tau\tau}$ = (2.67 $\pm$ 0.15)\%), the hadronic
efficiency in data is (98.32 $\pm$ 0.21)\%.  This can be compared with
a hadronic efficiency in Monte Carlo of (98.54 $\pm$ 0.22)\%, a
difference $+$ uncertainty in quadrature of 0.38\%.

%---------------------------------------------------------------------
%
\subsubsection{Monte Carlo Efficiency Studies}
%
%---------------------------------------------------------------------

Hadronic Monte Carlo is used to translate cascade $\Upsilon(1S)$
efficiencies to stationary $\Upsilon(1S)$ efficiencies (where the
$\gamma$ = 1.003 boost and potential track confusion from the pions
are removed), and to measure $\Upsilon(2S)$ and $\Upsilon(3S)$
efficiencies.  It is divided into four decay modes: $ggg$, $gg\gamma$,
$q\bar{q}$, and cascades to lower $b\bar{b}$ ($\Upsilon$ or $\chi_b$)
states.

Monte Carlo for the three resonances yielded the same efficiency for
$ggg$ (99.0\%), for $gg\gamma$ (94.6\%), and for $q\bar{q}$ (96.4\%)
at the 0.2\% level.  Since these are all high efficiencies, the
aggregate
\begin{eqnarray}
  \epsilon_\subs{non-cascade} &=&
  (1-\mathcal{B}_{gg\gamma}-R\mathcal{B}_{\mu\mu}) 99.0\% +
  \mathcal{B}_{gg\gamma} 94.6\% + R\mathcal{B}_{\mu\mu} 96.4\% \\
  &=& (0.88) 99.0\% + (0.03) 94.6\% + (0.09) 96.4\% = 98.7\%
\end{eqnarray}
is insensitive to uncertainties in the branching fractions.  About
half of the $\Upsilon(2S)$ and $\Upsilon(3S)$ decays are to cascades,
which would raise the relative fraction of $q\bar{q}$ in the above by
a factor of two: this lowers $\epsilon_\subs{non-cascade}$ by only
0.25\%.

For $\Upsilon(1S)$, this $\epsilon_\subs{non-cascade}$ of (98.7 $\pm$
0.2)\% is the total hadronic efficiency, which agrees with the (98.54
$\pm$ 0.22)\% obtained from boosted Monte Carlo.  Therefore, the boost
and potential for track confusion in the $\Upsilon(2S) \to \pi^+\pi^-
\Upsilon(1S)$ study affects $\Upsilon(1S)$ by at most 0.3\%
(uncertainty in the stationary Monte Carlo is negligible).  This is a
contribution to the $\Upsilon(1S)$ systematic error only.

The efficiency of $\Upsilon(2S)$ and $\Upsilon(3S)$ cascades is (95.27
$\pm$ 0.10)\% and (95.90 $\pm$ 0.10)\%.  This lower efficiency is
entirely due to $e^+e^-$ and $\mu^+\mu^-$ in some final states, as
these pass the hadronic selection criteria under 1\% of the time (and
then, only due to final state radiation lowering the largest track
momentum), while the other final states have a 98.5\% efficiency
(reminiscent of $\epsilon_\subs{non-cascade}$ above).  Branching
fractions to lower $b\bar{b}$ resonances, $\mathcal{B}_{ee}$, and
$\mathcal{B}_{\mu\mu}$ are sufficiently well-measured that they add
only 0.1\% to the efficiency uncertainty.

The total hadronic efficiency for $\Upsilon(2S)$ and $\Upsilon(3S)$
from Monte Carlo is (97.0 $\pm$ 0.3)\% and (97.3 $\pm$ 0.3)\%.

%---------------------------------------------------------------------
%
\subsubsection{Conclusion: Efficiency Bottom Line}
%
%---------------------------------------------------------------------






%---------------------------------------------------------------------
%
\section{Luminosity Measurement}
%
%---------------------------------------------------------------------

%---------------------------------------------------------------------
%
\subsection{Identifying Gamgam ($\bf e^+e^- \to \gamma\gamma$) Events}
%
%---------------------------------------------------------------------

%---------------------------------------------------------------------
%
\subsection{Measuring Gamgam Trigger Efficiency with Bhabhas}
%
%---------------------------------------------------------------------

%---------------------------------------------------------------------
%
\subsection{Calibrating the Luminosity Measurement}
%
%---------------------------------------------------------------------

%---------------------------------------------------------------------
%
\section{Stability of the Cross-Section Measurement}
%
%---------------------------------------------------------------------

%---------------------------------------------------------------------
%
\subsection{Synchronization of the Drift Chamber and Calorimeter}
%
%---------------------------------------------------------------------

%---------------------------------------------------------------------
%
\subsection{Cross-Section at Each Beam Energy is Constant}
%
%---------------------------------------------------------------------

%---------------------------------------------------------------------
%
\section{Lineshape Fits}
%
%---------------------------------------------------------------------

%% \label{fig:scanfits}



%---------------------------------------------------------------------
%
\section{Conclusions}
%
%---------------------------------------------------------------------

%---------------------------------------------------------------------
%
\section{Acknowledgements}
%
%---------------------------------------------------------------------

We gratefully acknowledge the effort of the CESR staff in providing us
with excellent luminosity and running conditions.  This work was
supported by the National Science Foundation, and the U.S. Department
of Energy.

%---------------------------------------------------------------------
%
%\section{Figures}
%
%---------------------------------------------------------------------

%---------------------------------------------------------------------
%
%\section{Bibliography}
%
%---------------------------------------------------------------------
\def\endpoint{;~~}
\def\Journal#1&#2&#3(#4){#1{\bf #2}, #3 (#4)}
\def\NIM{Nucl. Instr. and Meth. }
\def\NIMA{Nucl. Instr. and Meth. A }
\def\NPB{Nucl. Phys. B }
\def\PLB{Phys. Lett. B }
\def\PRL{Phys. Rev. Lett. }
\def\PRD{Phys. Rev. D }
\newpage
\begin{thebibliography}{99}

\bibitem{cleo3det}     CLEO Collaboration, CLNS-94-1277; D.\ Peterson \etal, {\Journal\NIMA&478&142(2002)}

\bibitem{kb}     Primer on Onium Widths, but a copy with interference; K.\ Berkelman, ???

\end{thebibliography}

\end{document}




%% \begin{table}
%% \begin{center}
%% \caption{Features of Set A and Set B.}
%% \smallskip
%% \begin{tabular}{lcc}
%% \hline
%% Quantity & ~~~~Set A & ~~~~Set B\cr 
%% \hline\hline
%% Fraction of total \nbb\     & 55\%    & 45\%  \cr
%% Track Resolution            &         &       \cr
%% \hfil{`A' Coefficient}      & 0.0055  & 0.0044\cr
%% \hfil{`B' Coefficient ($\gev^{-1}$)}      & 0.0011  & 0.0010\cr
%% $K^+\pi^-$ Mode             &         &       \cr
%% \hfil{$\sigma_{\mb}$ (\mev)}  & 2.7     & 2.7   \cr
%% \hfil{$\sigma_{\de}$(\mev)}   & 22      & 19    \cr
%% \hfil{Efficiency}           & 38\%    & 45\%  \cr
%% $K^+\pi^0$ Mode             &         &       \cr
%% \hfil{$\sigma_{\mb}$ (\mev)}  & 3.1     & 3.1   \cr
%% \hfil{$\sigma_{\de}$(\mev)}   & 31      & 31    \cr
%% \hfil{Efficiency}           & 33\%    & 35\%  \cr
%% $\pi^0\pi^0$ Mode\footnote{Resolutions are given as average of low-side and high-side half-resolutions.}           &         &       \cr
%% \hfil{$\sigma_{\mb}$ (\mev)}  & 3.6     & 3.6  \cr
%% \hfil{$\sigma_{\de}$(\mev)}   & 43      & 43    \cr
%% \hfil{Efficiency}           & 22\%    & 22\%  \cr
%% \hline
%% \end{tabular}
%% \label{tab:oldnew}
%% \end{center}
%% \end{table}

%% \bibitem{fleischeretal} 
%% Y.-Y. Keum, H.-N. Li, and A.I. Sanda, arXiv:hep-ph/0201103; 
%% M. Neubert,                  {JHEP} {\bf 9902} (1999) 014;
%% M. Neubert and J.L. Rosner,  {\Journal\PRL&81&5076(1998)};

%% \bibitem{cleo3det}     CLEO Collaboration, CLNS-94-1277; D. Peterson \etal, {\Journal\NIMA&478&142(2002)}

%% \bibitem{pdg} Particle Data Group, {\Journal\PRD&66&010001(2002)}.\label{ref:pdg}
