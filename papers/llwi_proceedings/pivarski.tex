\documentclass{ws-procs9x6}

\begin{document}

\title{Di-electron Widths of the $\Upsilon(1S)$, $\Upsilon(2S)$, and
  $\Upsilon(3S)$ from the CLEO-III Detector
\footnote{\uppercase{T}his work is supported by the \uppercase{A}.\uppercase{P}.~\uppercase{S}loan \uppercase{F}oundation, the \uppercase{N}ational \uppercase{S}cience \uppercase{F}oundation, and the \uppercase{U}.\uppercase{S}.~\uppercase{D}epartment of \uppercase{E}nergy.}}

\author{Jim Pivarski}

\address{Cornell University, \\
Ithaca, NY 14853, USA \\
E-mail: mccann@lepp.cornell.edu}

\maketitle

\newcommand{\subs}[1]{{\mbox{\scriptsize #1}}}
\newcommand{\inv}{$^{-1}$}
\newcommand{\PM}{$\pm$}
\newcommand{\ups}{$\Upsilon$}
\newcommand{\gee}{$\Gamma_{ee}$}
\newcommand{\us}{$\Upsilon(1S)$}
\newcommand{\uss}{$\Upsilon(2S)$}
\newcommand{\usss}{$\Upsilon(3S)$}
\newcommand{\es}{$\epsilon_{1S}$}
\newcommand{\ess}{$\epsilon_{2S}$}
\newcommand{\esss}{$\epsilon_{3S}$}
\newcommand{\ee}{$e^+e^-$}
\newcommand{\mumu}{$\mu^+\mu^-$}
\newcommand{\tautau}{$\tau^+\tau^-$}
\newcommand{\gamgam}{$\gamma\gamma$}
%\newcommand{\ggg}{$ggg$}
\newcommand{\gggamma}{$gg\gamma$}
\newcommand{\qqbar}{$q\bar{q}$}
\newcommand{\bee}{${\mathcal B}_{ee}$}
\newcommand{\bmm}{${\mathcal B}_{\mu\mu}$}
\newcommand{\btt}{${\mathcal B}_{\tau\tau}$}
\newcommand{\bcas}{${\mathcal B}_\subs{cas}$}
\newcommand{\geehadtot}{$\Gamma_{ee}\Gamma_\subs{had}/\Gamma_\subs{tot}$}
\newcommand{\twotoone}{$\Upsilon(2S) \to \pi^+\pi^- \Upsilon(1S)$}
\newcommand{\pipi}{$\pi^+\pi^-$}
\newcommand{\evis}{$\epsilon_\subs{vis}$}
\newcommand{\ecuts}{$\epsilon_\subs{cuts}$}
\newcommand{\ebeam}{$E_\subs{beam}$}
\newcommand{\ecm}{$E_\subs{CM}$}
\newcommand{\pmax}{$|\vec{p}_\subs{max}|$}
\newcommand{\visen}{$E_\subs{vis}$}
\newcommand{\dxy}{$d_\subs{XY}$}
\newcommand{\dz}{$d_\subs{Z}$}
\newcommand{\vtd}{$V_{td}$}
\newcommand{\twotrack}{{\tt two-track}}
\newcommand{\hadron}{{\tt hadron}}
\newcommand{\radtau}{{\tt rad-tau}}
\newcommand{\eltrack}{{\tt $e^\pm$-track}}
\newcommand{\barrelbhabha}{{\tt barrel-bhabha}}
\newcommand{\axial}{{\tt AXIAL}}
\newcommand{\stereo}{{\tt STEREO}}
\newcommand{\cblo}{{\tt CBLO}}
\newcommand{\cbmd}{{\tt CBMD}}
\newcommand{\cbhi}{{\tt CBHI}}

\abstracts{We determine the di-electron widths of the \us, \uss, and
  \usss\ resonances with better than 2\% precision by integrating the
  cross-section of $e^+e^- \to \Upsilon$ over the \ee\ center-of-mass
  energy.  Using \ee\ scans of the \ups\ resonances at the Cornell
  Electron Storage Ring and measuring \ups\ production with the CLEO
  detector, we find di-electron widths of
1.252 \PM\ 0.004 ($\sigma_{\mbox{\scriptsize stat}}$) \PM\ 0.019 ($\sigma_{\mbox{\scriptsize syst}}$) keV,
0.581 \PM\ 0.004 \PM\ 0.009 keV, and
0.413 \PM\ 0.004 \PM\ 0.006 keV for the \us, \uss, and \usss,
respectively.
}

\section{Introduction and Motivation}

The \ups\ meson is a bottom quark and an anti-bottom quark, Strongly
bound in a vector state labeled by excitation number ($1S$, $2S$, and
$3S$), and its di-electron width, \gee, is the decay rate of the \ups\
into \ee.  This width is a basic parameter of each \ups\ state, as it
characterizes the spatial size of the meson.  Non-perturbative QCD is
required to calculate \gee, so experimental measurements of \gee\ test
non-perturbative techniques.

Recent advances in Lattice QCD, in which quark and gluon fields are
represented on a lattice in a computer, have made few-percent
calculations of many quantities\cite{unquenched}.  Di-electron width
calculations of the \us, \uss, and \usss\ put extreme demands on the
continuum limit, as \gee\ is proportional to the quark wavefunction at
the origin.  To test these calculations, we have measured \gee\ for
the first three \ups\ resonances with 1--2\% precision.

Agreement between Lattice calculations and our measured \gee\ would
lend credence to calculations of similar quantities.  A quantity of
particular interest is the $B$ meson decay constant, $f_B$, which
obfuscates measurements of the CKM element $V_{td}$ because it is
poorly known.  If an $f_B$ calculation may be trusted within a few
percent of itself, our present knowledge of $V_{td}$ would shrink from
20\% to a few percent.

\section{Measurement Method}

We determine the $\Upsilon \to e^+e^-$ decay rate from the $e^+e^- \to
\Upsilon$ production cross-section:
\begin{equation}
  \Gamma_{ee} = \frac{{M_\Upsilon}^2}{6\pi^2} \int \sigma(e^+e^- \to
  \Upsilon) \, dE_\subs{CM}
\end{equation}
where $M_\Upsilon$ is the \ups\ mass and \ecm\ = $\sqrt{s}$ is the
center-of-mass energy of colliding \ee\ beams.  We integrate by
sampling the cross-section at several \ecm\ values near the \ups\
resonance peak and fitting to a parameterized function whose integral is
known.  Our fit function is a Breit-Wigner resonance peak convoluted
by a Gaussian to simulate beam energy spread, also convoluted by an
initial state radiation distribution\cite{kf}.  Additional terms
account for backgrounds, including interference between the resonance
peak and the continuum backgrounds for \qqbar\ and \tautau\ final
states.

To measure the \ups\ production cross-section, we count hadronic
events and divide by integrated luminosity, where we define a hadronic
event to be any beam-beam collision that results in a final state
other than \ee, \mumu, and \tautau.  A well-measured
fraction\cite{istvan} of \ups\ mesons decay into \ee, \mumu, and
\tautau\ (about 7\%), so we use this fraction to correct our
cross-section measurement.

\subsection{Backgrounds}

Depending on \ecm, 30\% to 100\% of our hadron counts are not \ups\
events.  These backgrounds have a flat dependence on \ecm, whereas the
signal is highly peaked (see Figure~\ref{background}), so signal and
background are distinguished by the fit function.  The majority of the
backgrounds are continuum processes which depend on \ecm\ as $1/s$,
and 8\% of the continuum (at 9~GeV) are two-photon fusion events
($e^+e^- \to e^+e^- X$) which have a $\log s$ dependence.  Radiation
to lower-energy resonances contributes at the $\sim\frac{1}{2}$\%
level.

\begin{figure}
\centerline{\epsfxsize=0.86\linewidth \epsfbox{pivarski_fig1.eps}}
\caption{The \usss\ resonance peak, presented in log scale to
  highlight backgrounds.  ``Raw cross-section'' is hadronic event
  yield per nb\inv, which includes backgrounds, but not efficiency
  corrections or corrections for \ee, \mumu, and \tautau\ final
  states. \label{background}}
\end{figure}

Cosmic rays, collisions between beam particles and gas (beam-gas), and
collisions between beam particles and the wall of the beampipe
(beam-wall) are not proportional to integrated luminosity, so they
must be subtracted from the hadron counts.  We determine the number of
cosmic rays at each \ecm\ by normalizing to a control sample acquired
with no colliding beams, and similarly estimate beam-gas and beam-wall
with single-beam samples.

\subsection{Efficiency}

We determine the efficiency of our triggers and event selection for
\us\ events without assuming a decay model for the \ups\ meson.  From
our 1.3~fb\inv\ \uss\ sample, we select \twotoone\ events by requiring
the \pipi\ to recoil against an object with the \us\ mass.  This way,
we trigger and select events independently of way the \us\ decays,
leaving the set of \us\ decays unbiased.  Applying our triggers and event
selection to the \us\ decays reveals a hadronic efficiency of (97.8 \PM\
0.5)\%.  We extrapolate this efficiency from the \us\ to the \uss\ and
\usss\ using Monte Carlo and measurements of $\Upsilon(nS) \to X
\Upsilon(mS) \to X \mu^+\mu^-$ branching fractions.  This procedure is
discussed in greater detail elsewhere\cite{prl}\mbox{ }\cite{thesis}.

\subsection{Integrated Luminosity}

To measure integrated luminosity, we count Bhabha events ($e^+e^- \to
e^+e^-$) and divide by the efficiency-weighted Bhabha cross-section,
determined from Monte Carlo simulations\cite{babayaga}\mbox{
}\cite{mc}.  We evaluate the systematic uncertainty in the
efficiency-weighted cross-section (1.3\%) by comparing Bhabha-derived
integrated luminosities with values determined from $e^+e^- \to
\mu^+\mu^-$ and $e^+e^- \to \gamma\gamma$, following the method used
in CLEO-II\cite{oldlumi}.

\subsection{Beam Energy Uncertainty}

The potential effect of fluctuations or drifts in the calibration of
the beam energy measurement was minimized by splitting the data-taking
into short, 10-hour scans, and by alternating measurements above and below
the resonance peak.  The point of maximum slope
($d\sigma/dE_\subs{CM}$) was measured twice to bound fluctuations in
the calibration through their effect on cross-section between the
two measurements, leading to a 0.2\% uncertainty in \gee.

\section{Fit Results}

The \us, \uss, and \usss\ fits are presented in Figure~\ref{fits},
which yield the following values for \gee.  They are consistent with
and more precise than the current world averages\cite{pdg}.
\begin{center}
\begin{tabular}{|c c|}
\hline
$\Gamma_{ee}(1S)$ & 1.354 \PM\ 0.004 \PM\ 0.020 keV \\
$\Gamma_{ee}(2S)$ & 0.619 \PM\ 0.004 \PM\ 0.010 keV \\
$\Gamma_{ee}(3S)$ & 0.446 \PM\ 0.004 \PM\ 0.007 keV \\
\hline
\end{tabular}
\end{center}

\begin{figure}
\centerline{\epsfxsize=\linewidth \epsfbox{pivarski_fig2.eps}}
\caption{The \ups\ lineshape fits used to determine \gee.  Points with
  errorbars are data, the solid line is the fit, the dashed line
  represents backgrounds, and the pull (fit residual divided by
  uncertainty) of every point is presented above the plots.
  Measurements 100~GeV, 60~GeV and 45~GeV above the three resonance
  maxima are shown in insets. \label{fits}}
\end{figure}

Uncertainty in the hadronic efficiency (0.5\%) and in the overall
luminosity scale (1.3\%) dominate the systematic uncertainty, but they
are also shared by all three resonances and therefore cancel in ratios
of \gee.  The systematic uncertainty in
$\Gamma_{ee}(mS)/\Gamma_{ee}(nS)$ is 0.9\%--1.0\%.

\section{Conclusions}

Though Lattice QCD calculations of \gee\ have not been corrected for
lattice renormalization yet\cite{lattice}, this effect largely cancels
in the ratio of $\Gamma_{ee}(2S)$ to $\Gamma_{ee}(1S)$.  The Lattice
calculation compares favorably with our measurement (see
Figure~\ref{lattice}), though its uncertainty is 10\% due to the steep
dependence on lattice spacing size.  We eagerly await the full
calculations.

\begin{figure}
\centerline{\epsfxsize=0.86\linewidth \epsfbox{pivarski_fig3.eps}}
\caption{Lattice calculations of $\Gamma_{ee}(2S)/\Gamma_{ee}(1S)$
  with our result overlaid.  \label{lattice}}
\end{figure}

\begin{thebibliography}{0}

\bibitem{unquenched}
C.T.H.~Davies {\it et al.}  (HPQCD Collaboration),
{\it Phys.\ Rev.\ Lett.}  {\bf 92}, 022001 (2004).

\bibitem{kf}
E.A.~Kuraev and V.S.~Fadin,
{\it Sov.\ J.\ Nucl.\ Phys.}  {\bf 41}, 466 (1985)
[Yad.\ Fiz.\  {\bf 41}, 733 (1985)].

\bibitem{istvan}
G.S.~Adams {\it et al.}  (CLEO Collaboration),
{\it Phys.\ Rev.\ Lett.}  {\bf 94}, 012001 (2005).

\bibitem{prl}
J.L.~Rosner {\it et al.},
{\it Phys. Rev. Lett.} {\bf 96}, 092003 (2006).

\bibitem{thesis}
J.~Pivarski, Cornell University, Ph.D.~thesis, {\tt hep-ex/0604026} (2006).

\bibitem{babayaga}
C.M.~Carloni~Calame {\sl et al.},
{\it Nucl. Phys. Proc. Suppl. B} {\bf 131}, 48 (2004).

\bibitem{mc}
R.~Brun {\it et al.}, {\textsc Geant} 3.21, CERN Program Library Long
Writeup W5013 (1993), unpublished.

\bibitem{oldlumi}
G.D.~Crawford {\it et al.} (CLEO Collaboration),
{\it Nucl. Instrum. Methods Phys. Res., Sect A} {\bf 345}, 429 (1994).

\bibitem{pdg}
S.~Eidelman {\it et al.}  (Particle Data Group),
Phys.\ Lett.\ B {\bf 592}, 1 (2004).

\bibitem{lattice}
A.~Gray, I.~Allison, C.~T.~H.~Davies, E.~Gulez, G.~P.~Lepage, J.~Shigemitsu and M.~Wingate,
{\it Phys.\ Rev.\ D} {\bf 72}, 094507 (2005).

\end{thebibliography}

\end{document}
